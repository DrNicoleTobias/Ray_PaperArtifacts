%%%%%%%%%%%%%%%%%%%%%%%%%%%%%%%%%%%%%%%%%%%%%%%%%%%%%%%%%%%%%%%%%%%%%%%%%%%%%%%%
% Template for JSys papers.
%
% History:
%
% - TEMPLATE for the Journal of Systems Research, prepared for submissions to
%   JSys in 2020 by Vijay Chidambaram (University of Texas at
%   Austin), Romain Jacob (ETH Zurich), and Vishwanath Seshagiri (Emory).
%   This was originally the USENIX template,
%   which was adapted with USENIX permission. Thanks!  
%
% - TEMPLATE for Usenix papers, specifically to meet requirements of USENIX '05.
%   originally a template for producing IEEE-format articles using LaTeX.
%   written by Matthew Ward, CS Department, Worcester Polytechnic Institute.
%   adapted by David Beazley for his excellent SWIG paper in Proceedings, Tcl
%   96. turned into a smartass generic template by De Clarke, with thanks to
%       both the above pioneers. Use at your own risk. Complaints to /dev/null.
%       Make it two column with no page numbering, default is 10 point.
%
% - Munged by Fred Douglis <douglis@research.att.com> 10/97 to separate the .sty
%   file from the LaTeX source template, so that people can more easily include
%   the .sty file into an existing document. Also changed to more closely follow
%   the style guidelines as represented by the Word sample file.
%
% - Note that since 2010, USENIX does not require endnotes. If you want foot of
%   page notes, don't include the endnotes package in the usepackage command,
%   below.
%
% - This version uses the latex2e styles, not the very ancient 2.09 stuff.
%
% - Updated July 2018: Text block size changed from 6.5" to 7"
%
% - Updated Dec 2018 for ATC'19:
%
%   * Revised text to pass HotCRP's auto-formatting check, with
%     hotcrp.settings.submission_form.body_font_size=10pt, and
%     hotcrp.settings.submission_form.line_height=12pt
%
%   * Switched from \endnote-s to \footnote-s to match Usenix's policy.
%
%   * \section* => \begin{abstract} ... \end{abstract}
%
%   * Make template self-contained in terms of bibtex entires, to allow this
%     file to be compiled. (And changing refs style to 'plain'.)
%
%   * Make template self-contained in terms of figures, to allow this file to be
%     compiled. 
%
%   * Added packages for hyperref, embedding fonts, and improving appearance.
%
%   * Removed outdated text.
%
%%%%%%%%%%%%%%%%%%%%%%%%%%%%%%%%%%%%%%%%%%%%%%%%%%%%%%%%%%%%%%%%%%%%%%%%%%%%%%%%

\newif\ifsubmit
%\submitfalse
\submittrue           % uncomment this line for submission.


\newcount\usepng\usepng=1               % use PNG versions of figures (smaller file)


\documentclass[letterpaper,twocolumn,10pt]{article}
\usepackage{jsys}
%\usepackage{jsys_camera_ready}

\usepackage{balance} % For balanced columns on the last page
\usepackage[normalem]{ulem}


\usepackage{graphicx}  % for \includegraphics


\ifsubmit
\usepackage[final]{fixme}
\else
\usepackage[draft]{fixme}
\fxsetup{
    author=,
    layout=inline,
    theme=colorsig
}
\fi
\newcommand{\xxx}{\fxwarning}


\usepackage{url}
\def\UrlBreaks{\do\/\do-}


\usepackage{siunitx}
%\usepackage{SIunits}


\ifnum\usepng=1
 \DeclareGraphicsExtensions{.png,.eps,.ps,.jpg}
\else
 \DeclareGraphicsExtensions{.ps,.eps,.png,.jpg}
\fi

\usepackage{ifpdf}
\ifpdf
\setlength{\pdfpagewidth}{8.5in}
\setlength{\pdfpageheight}{11in}
\else
\fi

\usepackage{microtype}
\usepackage{paralist}

%stuff for better looking tables
\usepackage{array}
\usepackage{booktabs}
\usepackage{multirow}
\usepackage{dcolumn}
\pdfcompresslevel=9       % highly compressed pdf

\newlength{\thinline}
\setlength{\thinline}{0.05em}
\newlength{\thickline}
\setlength{\thickline}{0.10em}
\newcommand{\ra}[1]{\renewcommand{\arraystretch}{#1}}
\newcolumntype{.}{D{.}{.}{-1}}

\usepackage{enumitem} % to fiddle with enumerate spacing

\usepackage{subcaption}
\usepackage{xspace}
% pretty captions: small text, bold figure title
%\usepackage[bf]{caption}
%\DeclareCaptionType{copyrightbox}
%\renewcommand{\captionfont}{\normalsize}


\newcommand{\headingg}[1]{\textbf{#1}}
\newcommand{\term}[1]{\emph{#1}}
\newcommand{\nowrapterm}[1]{\mbox{\emph{#1}}}
\newcommand{\secref}[1]{\mbox{Section~\ref{#1}}}
\newcommand{\plursecref}[1]{\mbox{Sections~{#1}}}
\newcommand{\symsecref}[1]{\mbox{\S~\ref{#1}}}
\newcommand{\plursymsecref}[1]{\mbox{\S\S~{#1}}}
\newcommand{\figref}[1]{\mbox{Figure~\ref{#1}}}
\newcommand{\tabref}[1]{\mbox{Table~\ref{#1}}}
\newcommand{\tblref}[1]{\tabref{#1}}
\newcommand{\algref}[1]{\mbox{Algorithm~\ref{#1}}}
\newcommand{\listingref}[1]{\mbox{Listing~\ref{#1}}}

% textual substitutions
\newcommand{\etal}{\mbox{et al.}}
\newcommand{\ie}{\textit{i.e.,}}
\newcommand{\eg}{\textit{e.g.,}}

%\newcommand{\degree}{\ensuremath{^\circ}}
\newcommand{\iv}{I--\kern-.08emV\xspace}

% No-indent paragraphs
\newcommand{\noind}[0]{\vspace{5 pt} \noindent}
\newcommand{\noindpar}[1]{\noind {\bf #1}}
%\newcommand{\noindpar}[1]{\noind \textbf{\small{#1}}}

\newcommand{\textoverline}[1]{$\overline{\mbox{#1}}$}

% Macros for notes, comments, and todo.
\ifsubmit
  \newcommand{\colorcomment}[3]{\relax}
  \newcommand{\greencomment}[2]{\relax}
  \newcommand{\yellowcomment}[2]{\relax}
  \newcommand{\cyancomment}[2]{\relax}
  \newcommand{\bluecomment}[2]{\relax}
  \newcommand{\redcomment}[2]{\relax}
  \newcommand{\magentacomment}[2]{\relax}

  \newcommand{\hey}[1]{\relax}
  \newcommand{\bibtex}[1]{\relax}
  \newcommand{\task}[1]{\relax}
  \newcommand{\question}[1]{\relax}
  \newcommand{\note}[1]{\relax}

\else
  % comment commands
  % comments should be signed at the end with the initials of the comment writer.
  \newcommand{\colorcomment}[3]{\par\noindent\textcolor{#1}{[{#2}: {#3}]}\par} % general case
  \newcommand{\greencomment}[2]{\colorcomment{green}{#1}{#2}} % hard to read
  \newcommand{\yellowcomment}[2]{\colorcomment{yellow}{#1}{#2}} % hard to read
  \newcommand{\cyancomment}[2]{\colorcomment{cyan}{#1}{#2}} % hard to read
  \newcommand{\bluecomment}[2]{\colorcomment{blue}{#1}{#2}}
  \newcommand{\redcomment}[2]{\colorcomment{red}{#1}{#2}}
  \newcommand{\magentacomment}[2]{\colorcomment{magenta}{#1}{#2}}

  \newcommand{\hey}[1]{\textcolor{magenta}{[{#1}]}} % quick inline comment
  \newcommand{\note}[1]{\magentacomment{Note}{#1}} % a paragraph comment.
  \newcommand{\bibtex}[1]{\magentacomment{@bibtex}{#1}} % a reference that should go to bibtex
  \newcommand{\task}[1]{\redcomment{Task}{#1}} % task the person with @name in the argument
  \newcommand{\question}[1]{\bluecomment{Q?}{#1}} % ask a question
\fi

\usepackage{listings}
\usepackage{color}

\definecolor{mygreen}{rgb}{0,0.6,0}
\definecolor{mygray}{rgb}{0.5,0.5,0.5}
\definecolor{mymauve}{rgb}{0.58,0,0.82}
\definecolor{myred}{rgb}{0.58,0,0.0}
\lstset{
    language=C,
         emph={when},
         basicstyle=\footnotesize\ttfamily,
         numberstyle=\tiny,
         numbersep=5pt,
         tabsize=2,
         extendedchars=true,
         breaklines=true,
                keywordstyle=\color{mymauve}\ttfamily\bfseries,
                emphstyle=\color{mymauve}\ttfamily\bfseries,
                stringstyle=\color{myred}\ttfamily,
                commentstyle=\color{mygreen}\ttfamily,
                morecomment=[l][\color{mygray}]{\#},
         showspaces=false,
         showtabs=false,
         %xleftmargin=17pt,
         %framexleftmargin=17pt,
         %framexrightmargin=5pt,
         framexbottommargin=4pt,
         escapeinside={\%*}{*)},
         showstringspaces=false
 }

%\usepackage{caption}
%\DeclareCaptionFont{white}{\color{white}}
%\DeclareCaptionFormat{listing}{\colorbox[cmyk]{0.43, 0.35, 0.35,0.01}{\parbox{\columnwidth}{\hspace{15pt}#1#2#3}}}
%\captionsetup[lstlisting]{format=listing,labelfont=white,textfont=white, singlelinecheck=false, margin=0pt, font={bf,footnotesize}}

%\setcopyright{acmcopyright} % if you give the rights to ACM
%\acmDOI{...} % DOI - Insert your DOI below...
%\acmISBN{...} % ISBN - Insert your conference/workshop's ISBN below...
%\acmYear{2023} % Insert Publication year
%\copyrightyear{2023} % Insert Copyright year (typically the same as above)
%\acmPrice{15.00}
%\acmConference[SenSys]{The 17th ACM Conference on Embedded Network Sensor Systems}{November 10-13, 2019}
%{New York, New York, USA}
%\acmJournal{TOSN}

\hyphenation{WISP RFID CRFID CRFIDs RF volt-age}
\newcommand{\sysnameraw}{Ray}
\newcommand{\sysname}{\sysnameraw\xspace}
\newcommand{\sysnames}{\sysnameraw{}s\xspace}
\newcommand{\numExp}{881\xspace} %updated by Nicole 1/30
\newcommand{\numDet}{881\xspace}
\newcommand{\numDir}{849\xspace} %updated by Nicole 1/30
\newcommand{\numDoors}{7\xspace}  %updated by Nicole 1/25
\newcommand{\numPeople}{9\xspace}  %updated by Nicole 1/25 - need to double check this
\newcommand{\SysAccuracy}{100\%\xspace} %updated by Nicole 1/25
\newcommand{\dirAccuracy}{96.4\%\xspace} %updated by Nicole 1/25
\newcommand{\minSeparation}{3-4 seconds\xspace}
\newcommand{\ITWdays}{64 days\xspace}
\newcommand{\ITWdeployedWaldos}{5\xspace}
\newcommand{\ITWpeopleEvents}{xxx\xspace}
\newcommand{\WaldoPWR}{wwPWR\xspace}
\newcommand{\EnoceanPWR}{eeePWR\xspace}


\usepackage{pifont}% http://ctan.org/pkg/pifont
\newcommand{\cmark}{\ding{51}}%
\newcommand{\xmark}{\ding{55}}%
%\newcommand{\syslongname}{\textbf{W}aldo, \textbf{A} \textbf{L}ow-power \textbf{D}oorjamb \textbf{O}ccupancy-tracker\xspace}
%\newcommand{\sysname}{LumiJamb\xspace}
%\newcommand{\syslongname}{\textbf{Lumi}nance based people tracking in the Door\textbf{Jamb}\xspace}
%\newcommand{\sysname}{SolarJamb\xspace}
%\newcommand{\syslongname}{\textbf{Solar} based people tracking in the Door\textbf{Jamb}\xspace}


%\nocaptionrule % This removes the ugly line above table captions on SIGPLAN templates

% Uncomment the following line if you want the columns of the last page equal in
% size. But note that doing so may cause issues with some document-generating
% tools.
% \usepackage{flushend}

% inlined bib file -> only for making the file compilable on its own.
%-------------------------------------------------------------------------------
\begin{filecontents}{\jobname.bib}
%-------------------------------------------------------------------------------
@Book{arpachiDusseau18:osbook, 
  author =       {Arpaci-Dusseau, Remzi H. and Arpaci-Dusseau Andrea C.}, 
  title =        {Operating Systems: Three Easy Pieces}, 
  publisher =    {Arpaci-Dusseau Books, LLC}, 
  year =         2015,
  edition =      {1.00}, 
  doi = 	{10.5070/SR31154815},}
@InProceedings{waldspurger02, 
  author =       {Waldspurger, Carl A.}, 
  title =  	{Memory resource management in {VMware ESX} server}, 
  booktitle =    {USENIX Symposium on Operating System Design and Implementation (OSDI)}, 
  year =   	2002, 
  pages =        {181--194}, 
  url =	{https://www.usenix.org/legacy/event/osdi02/tech/waldspurger/waldspurger.pdf}}
\end{filecontents}

% enable this for the camera-ready version
% \jsysfinaltrue


% for ORCID
\usepackage{orcidlink}
%-------------------------------------------------------------------------------
\begin{document}
%-------------------------------------------------------------------------------

%don't want date printed
\date{}

% make title bold and 14 pt font (Latex default is non-bold, 16 pt)
\title{\sysname: Batteryless Occupancy Monitoring using Reflected Ambient Light}

% for single author (just remove % characters)
% ORCID links to two nobel laureates for now, please change




\author{
  {\textsc{Nicole Tobias}}\orcidlink{0000-0003-4895-6792}\\
 Clemson University\\
  rtobias@clemson.edu
  \and
  {\textsc{Harsh Desai}}\orcidlink{}\\
  Carnegie Mellon University\\
harshd@andrew.cmu.edu
   \and
  {\textsc{Arwa Alsubh}}\orcidlink{0000-0003-3912-8143s}\\
  Umm AL-Qura University\\
  amsubhi@uqu.edu.sa
   \and
  {\textsc{Calvin Moody}}\orcidlink{}\\
  Facet Wealth\\
  calvinmcm@gmail.com
    \and
  {\textsc{Josiah Hester}}\orcidlink{}\\
  Georgia Institute of Technology\\
  josiah@gatech.edu
    \and
  {\textsc{Jacob Sorber}}\orcidlink{}\\
 Clemson University\\
jsorber@clemson.edu
} % end author


\maketitle

% Disable header and footer on the from page
\thispagestyle{empty}


% Disable header and footer on the from page
%\thispagestyle{empty}
\begin{abstract}
%The buildings of our science fiction dreams have always adapted to the needs of their occupants.
%Today, "smart buildings" are poised to become reality, enabled by advances in sensors that monitor room-level occupancy and movement.
Reliable and accurate room-level occupancy-tracking systems enable intelligent control of building functions like air conditioning and power delivery to adapt to the needs of their occupants.
%This allows buildings to be more capable of adapting to the needs of their occupants in their day-to-day activities and better optimize certain resources, such as power and air conditioning, to do so.
Unfortunately, existing occupancy-tracking systems are bulky, have short battery lifetimes, or are not privacy preserving.
Furthermore, retrofitting existing infrastructures with wired sensors is prohibitively expensive.

In this paper, we present \sysname, a \textit{batteryless}, room-level occupancy monitoring sensor that harvests energy from indoor ambient light reflections, and uses changes in these reflections to detect when people enter and exit a room.
%This information is then communicated by radio to a basestation for further processing and actuation.
\sysname is mountable at the top of a doorframe, allowing for detection of a person and the direction they are traveling at the entry and exit point of a room.
We evaluated the \sysname sensor in an office-style setting under mixed lighting conditions (natural and artificial) on both sides of the doorway with subjects exhibiting varying physical characteristics such as height, hair color, gait, and clothing.
We conducted \numExp controlled experiments in \numDoors doorways with \numPeople individuals and achieved a total detection accuracy of \SysAccuracy and movement direction accuracy of \dirAccuracy.  
We also describe an uncontrolled in-the-wild experiment using \sysname sensors across \ITWdays collectively in three different locations.
%Further, it judged the direction of movement correctly with an accuracy of \dirAccuracy.
%This paper also explores and discusses various practical factors that can impact the performance of the current system in actual deployments.
%Finally, we evaluate the power consumption of the \sysname device, demonstrating it's low-power draw.
%While challenges remain, t
\sysname demonstrates that ambient light reflections provide both a promising low-cost, long-term sustainable option for monitoring how people use buildings and an exciting new research direction for \textit{batteryless} computing.

\end{abstract}
\begin{figure}[t]
\centering
\includegraphics[width=\columnwidth]{figs/scenario.png}
\caption{ The overall system concept of \sysname, a batteryless, energy-harvesting, doorjamb mounted occupancy tracking and person detection enabling system.  This system uses reflective indoor lighting to both power the system and detect person entry and exit activity to a room.\label{fig:syspic}}
\end{figure}

\section{Introduction}
\label{sec:intro}

% intro to occupancy tracking
Understanding how occupants move, work, and live within a workplace or residence is essential for enabling health, efficiency, and security applications in smart buildings.
Appliances, computers, lighting, and heating and cooling systems can adapt their behavior depending on the number of occupants, their needs, and the context of their interactions.
Smart buildings can automatically identify indoor traffic patterns, poorly-used space, and congested walkways, generating data that can be used to further understand how people interact with buildings and the different spaces within them.
These benefits are dependent on the data that sensors collect as they observe people moving through a building and their interactions within the space.

Unfortunately, current occupancy-tracking systems are too large and costly to be considered for large-scale deployment, and are too high-maintenance for long-term monitoring.
There are also privacy concerns for occupancy tracking systems that can identify people's daily habits within a building, possibly without their knowledge.
Existing systems use ultrasound\cite{hnat2012doorjamb}, images\cite{tyndall2016occupancy, teixeira2007lightweight}, wearables\cite{fishkin2005hands}, instrumented objects\cite{buettner2009activity}, structural vibrations\cite{pan2016occupant}, and opportunistic data leaked from existing meters and security systems\cite{yangoccupancy2014}.
%\fxnote{[Can we make the above two sentences more crisp? They're too long and we might miss the point if they're inedible - HD]}
Some of these solutions~(like imaging) gather identifiable information.
Others require building remodeling, force users to change their behavior, or require structural models of the building.
For any of these solutions to work, we must either provide wired power to our sensors (which is usually expensive), or use batteries which increase cost, environmental impact, and fire risk, not to mention required replacement every few years~(even rechargeables).

% our solution
In this paper we present \sysname (overview shown in \figref{fig:syspic}), an occupancy monitoring sensor that is low-cost and low-maintenance, preserves occupant privacy, and can operate for decades\footnote{Actual lifetimes depend on environmental conditions, enclosure quality, and rates of decay for silicon and other circuit materials. The point is that the usual bottleneck (the battery) is not there. Lifetimes of 10--50 years are realistic but not guaranteed.} without wired power or batteries.
%

% why batteryless?
%Importantly, \sysname is batteryless to enable the long term deployment at scales required to instrument entire commercial buildings and residential neighborhoods.
%Batteries are expensive, bulky, environmentally unsustainable, and do not have long lifetimes (even rechargeables).
%By leaving the batteries behind, the device power supply becomes intermittent, complicating execution, data collection, energy management, and timekeeping.
%However, the ability for long term deployments at the scales envisioned for the Internet-of-Things provides enough motivation.

Like the UVa doorjamb sensor~\cite{hnat2012doorjamb}, \sysname is attached to a doorjamb and monitors movement in and out of the doorway.
Unlike previous solutions, \sysname does not use active sensors~(like ultrasonic range finders), but instead senses movement using the same ambient and florescent light reflections that power the sensor.
\sysname harvests solar energy from indoor lights to power all operations, and uses a combination of hardware and software techniques to detect human movement and direction as solar energy availability changes.
%Besides harvesting solar energy, \sysname processes in hardware and software the signal generated by an array of solar panels to detect when people enter or exit a room.
%When enough energy is available, and a person is walking through the doorway, \sysname uses ultrasonic range finders to measure the height of the person for identification.
%\sysname stores this information on device, and opportunistically broadcasts occupancy information to the basestation over a radio.


%Because of the incredibly tight energy constraints and unpredictable energy-harvesting conditions, \sysname uses approximate computing concepts to provide value to the global system no matter the energy conditions.
%\sysname dynamically adapts its duty cycle, task schedule, and data-transmission granularity, depending on the energy available.
%If relatively high amounts of energy are available in the environment, then \sysname will use the ranging sensor to get heights for each person passing through the doorway, and will send each passing event over the air through BLE to the basestation.
%If relatively low amounts of energy are available in the environment, then \sysname will turn off the range finder, and use only the passive, zero-energy-cost solar panels to detect passing events.
%Instead of transmitting on each passing event in this low energy state, \sysname instead sends global statistics opportunistically, detailing the number of events or people that have passed since the last interaction.


\subsection*{Contributions}

The contributions of this paper include:

\begin{compactenum}
	\item A novel system design for unobtrusive, long-term, low-cost, zero-maintenance occupancy tracking.
	\item An in-depth analysis of the design considerations for batteryless, intermittently powered continuous sensing systems that have computation and data with high temporal locality that can be broadly applied to other batteryless sensing applications.
	%\item A novel approximate computing method for adapting the duty cycle of batteryless occupancy  sensors depending on energy available, trading off accuracy for energy.
%\fxnote{[We need to reword this - HD]} %  1) record heights, 2) then record each event and report, 3) only report stats of people going in and out
	%\item An investigation of security and privacy considerations stemming from ubiquitous, low cost occupancy sensors.
	\item An implementation, deployment, and evaluation of \sysname that explores the strengths and limitations of our methods.
\end{compactenum}

\noind
\sysname is, to our knowledge, the first batteryless occupancy monitoring system, and demonstrates the potential and usefulness of long-lived, energy-harvesting, batteryless sensing operation in the built environment.
In this paper we present our design, a working prototype, and evaluation results showing efficacy of the approach.


% These types of condensed table of contents in papers communicate essentially nothing, let's not waste the space - JDH
%\noind
%In the following sections, we give background and related works on occupancy detection and batteryless sensing (\secref{sec:background}), outline the \sysname system design (\secref{sec:system}) and implementation (\secref{sec:implementation}), demonstrate the feasibility and general applicability of \sysname in a variety of situations (\secref{sec:evaluation}), discuss limitations, privacy considerations, and future work (\secref{sec:discussion}), and conclude (\secref{sec:conclusions}).

\section{Batteryless People Sensing}
\label{sec:background}

% Batteryless sensing is needed for large scale because....
Energy-harvesting batteryless sensors are critical to an affordable and sustainable Internet-of-Things~(IoT) and the future of smart buildings.
%
Running wires to power new sensors and other devices is expensive and not always feasible.
On the other hand, batteries are expensive, bulky, and often hazardous.
Even rechargeable batteries wear out after a few years, and replacing trillions of additional batteries every year would be both expensive and irresponsible.
%
In contrast, batteryless sensors powered entirely with harvested energy cost less, weigh less, and can operate for decades with less impact on the environment.

%Long term deployments at scale, like \sysname,
However, batteryless sensing is challenging.
Energy is stored in one or more small, cheap capacitors to improve efficiency and responsiveness~\cite{jhester:ufop:sensys}.
Harvested energy is variable and difficult to predict.
Power failures are common, interrupting computation and data processing, sensing, and communication.
Clocks reset and volatile memory is lost frequently, complicating a developer's ability to build robust and sophisticated applications.

Recent advances in checkpointing~\cite{ransford2011mementos, balsamo2015hibernus}, consistent execution~\cite{colin2016chain, Lucia:2015:Dino}, timekeeping~\cite{hester2016persistent}, energy management~\cite{jhester:ufop:sensys}, testing~\cite{ekho-sensys}, and debugging~\cite{colin_edb} address key challenges, and have enabled new and interesting applications: tracking building and appliance energy consumption~\cite{debruin2013monjolo,campbell2014energy} and monitoring greenhouses~\cite{jhester:ufop:sensys}.


In spite of these improvements, current batteryless sensing applications are limited and typically fall into one of two categories: those that depend on an RFID reader and those that opportunistically detect valid, useful data whenever measured. 
Power failures and long outages makes it difficult or impossible to gather streams of uninterrupted data, and provide high quality of service to the user.
This has led to avoidance of some sensing applications that work best with uninterrupted sensing; such as occupancy monitoring. 

Occupancy monitoring applications instrument buildings, people, or other technology, to get a better understanding of the number of people in a room.
This information is the baseline data for successful operation of smart building functions; such as intelligent temperature and HVAC control, efficiency monitoring, elderly tracking, and other applications.
Existing occupancy monitoring systems use many sensing techniques and deploy in many different form factors, with doorway based sensing being one promising  method~\cite{hnat2012doorjamb, sonicdoor-buildsys2017}.
In this paper, we investigate the challenges of occupancy monitoring using intermittently powered devices mounted in doorways.
We recognize three major challenges to implementing a successful system:
% What are the challenges for batteryless occupancy sensing beyond the regular challenges?

\noindpar{Intermittent operation:}
The effect of small energy storage, unpredictable energy harvesting means that occupancy sensing devices must be careful to (1) manage energy to reduce power failures (so as not to miss people walking through the door), (2) use ultra low power sensing techniques and passive methods to gather signal and support the applications, and (3) be failure resistant, gracefully handling power failures and returning to deterministic states.

\noindpar{Signal from Energy harvesting noise:} 
Door-mounted occupancy sensors can harvest energy from indoor and ambient lighting using solar panels pointed towards the floor or other reflective surfaces.
This energy is readily available in typical residential, industrial, and commercial buildings.
This energy can be harvested, stored, and used to power sensing tasks.
Importantly, this energy is also a \textit{signal} that can be processed to gain insight into the changing environment of the building, the movement of people and objects, or even the time of day.
This correspondence between the energy that powers the sensor and the data that makes the application work can be leveraged to enable occuapcny detection.
If a door-mounted entry and exit sensor has solar panels that point down towards the floor, a person walking through the doorway would occlude the light, lowering the energy harvested for that point in time.
This event could be tracked passively, as the solar panels themselves are free sensors.
However, this signal is noisy, and the resolution and magnitude of signal depends on the behavior of the sensor~\cite{ekho-sensys}.
Processing useful signal from energy harvesting noise under a constrained computational and energy situation poses challenges at the hardware and firmware level for batteryless occupancy monitoring.

\noindpar{Human and building confounds:}
Harvesting both energy and signal from solar panels introduces confounding factors from the variability of lighting in buildings, and the variability of people and their habits.
Many buildings will have some well lit rooms bordering dim hallways, or vice-versa. 
Other rooms may have an abundance of natural light, while some have only artificial light.
Peoples clothing, hair color, skin color, walking speed, and height will all affect the amount of occluded or reflected light and potentially change the readings on the solar panel.
Any system that promises robust occupancy monitoring using energy harvesting must be able to handle with these many confounding factors.





% We are not doing any of this now
%\noindpar{Approximate Tasks:} Continuous sensing application require uninterrupted streams of data to ensure no events are missed.
%For occupancy setting, it is observed that it is much better to enable longer streams of uninterrupted sensing at a lower application quality or application accuracy than to gather short, intermittent bursts at a higher quality, requiring more energy, and constraining execution to only a certain level of available energy.
%To support nearly continuous sensing, approximate computing can be leveraged, trading accuracy and quality of services for uninterrupted operation.
%The following are required for this to work: 1)~identifying the levels of service that can be supported by the application, and 2)~knowing when to switch between these levels of service.
%Approximate computing applied to batteryless systems works especially well when sensor data exhibits high temporal locality, meaning that a data point gathered immediately after another may be worthless, as nothing has changed (for example when monitoring an empty doorway).
%Knowing when to reduce the level of service is a key challenge.


% Introduce and bridge to the next section (system design that answers these challenges)
Batteryless occupancy sensing has never been done; but can take advantage of a key observations to provide reliable service---the reality that the applications' data stream (energy harvesting from solar panels) can also be harvested and used as energy that powers the device.
By taking advantage of the temporal locality of energy harvesting and data in occupancy sensing, we can build a long-lived sensor that tracks and identifies people as they enter and exit rooms.
In the following sections we discuss \sysname, a novel sensing platform that demonstrates the feasibility and utility of energy harvesting, intermittently powered devices, for sensing in the sustainable future Internet-of-Things.


\section{\sysname}
\label{sec:system}
% Basic what it is
\sysname is a slim, batteryless, occupancy-monitoring sensor system mounted to the top of a doorframe.
It is powered by energy harvested from two arrays of indoor solar panels pointed at the floor.
The panels serve two roles: 1)~energy harvester and 2)~sensor.
These panels gather \textbf{energy} for computation, sensing, and signaling while also providing the \textbf{signal} that \sysname uses to detect when a person walks through the doorway in the form of variations in the harvested energy.
\sysname records the direction---entry or exit---of each doorway event and stores this information in non-volatile memory for later transmission.


\noindpar{Design Goals:} Unpredictable power availability coupled with confounding factors of human-based sensing make designing an intermittently powered occupancy sensor challenging.
We designed \sysname to meet the following design goals which address specific challenges:
\begin{enumerate}
	\item \textbf{Availability:} Doorway events can occur at any time.
	While many intermittent sensors are able to gather data opportunistically as energy is available, \sysname is designed to conserve its harvested energy so that it is available to detect ephemeral doorway events, whenever they occur.
	\item \textbf{Accurate direction:} In addition to detecting someone passing through the doorway, \sysname uses angled solar panels to accurately determine their direction.
	This plays a crucial role in inferring the occupancy of rooms and buildings.
	\item \textbf{Variable lighting conditions:} Indoor lighting conditions can change over time, due to human behavior and the relative movement of the sun.
	We have designed \sysname to work in a range of different lighting conditions by using detection circuits that respond to changes in light level, independent of the absolute amount of light, as well as tuning mechanisms built into the prototype.
	\item \textbf{Variable human characteristics:} An effective occupancy sensor should work well in spite of variations in clothing, hair, height, walking speed, and skin color. By focusing on changes in total reflected light, \sysname is robust to these human variations.
	\item \textbf{Form factor:} We want \sysname to be easy to deploy, to fit unobtrusively inside a door frame, and avoid contact with doors (on frames with doors).
	We could harvest more energy by wrapping \sysname around the doorframe, but the system would be more expensive, harder to deploy, and more likely to interfere with doors, while also changing the aesthetics of the doorway.
\end{enumerate}

\noindpar{What \sysname is not.}
We also want to be clear about what \sysname is \emph{not}.
\sysname is \emph{not} a security device.
\sysname helps building owners and managers understand how people move through buildings, but it is \emph{not} designed to thwart malicious behavior.
We can easily trick \sysname with a flashlight or reflective materials, and we can disable it completely by covering its solar panels or turning off the lights.
Users looking to prevent shenanigans or tomfoolery should use a different device.
Users looking for a long-lived, low-maintenance, best-effort batteryless occupancy sensor for monitoring normal behaviors should read on.

% fig refs
An overview of the \sysname architecture is shown in \figref{fig:overview} and our \sysname prototype device is shown in \figref{fig:prototype}.
% Intro the rest of the section
We detail our approach to meeting these design goals and answering their associated challenges in the rest of the section~--- specifically we describe the \sysname architecture and design, the detection mechanism, and the energy management operations.


\begin{figure}[t]
\centering
\includegraphics[width=0.75\columnwidth]{figs/overview.pdf}
\caption{The \sysname architecture overview. \sysname uses the energy and signal from two sets of solar panels to both power the sensor and detect people passing into and out of a doorway. Two detector circuits each monitor half of the solar panels mounted in series that face inward, and outward in the doorway. On detection, the detectors wake up the MCU to process, log, or communicate occupancy information.  \label{fig:overview}}
\end{figure}


\subsection{Energy Harvesting and Management}
\sysname takes advantage of the ubiquity of indoor light in homes and offices.
Solar panels are mounted to the top of the door frame, pointing down toward the floor---half tilted \ang{20} inward and half tilted \ang{20} outward.
Pointing the panels downward is not ideal for energy harvesting but effective for detecting doorway events and provides a slim, easy-to-deploy form factor.
The \ang{20} tilt helps \sysname determine walking direction, as a person will affect one half of the panels before the other.

To maximize energy harvesting, we connect the two sets of solar panels---the inward-facing set and the outward-facing set---in series.
A series configuration conveniently combines the two panel sets into a single power source, but we can't directly measure the raw voltage on each set since the sets lack a common reference ground\footnote{For a series connection, we connect the positive terminal of the first panel set to the negative terminal of the second.}.
Instead, we measure the voltage of the outward-facing set alone, and the combination of the two sets.
We could compute the inward panels' voltage by subtracting the two; however, we have found that we can skip this step and just compare the two measurements directly, as shown in \figref{fig:traces}, to determine walking direction.

%Therefore, we feed the signal from one set of solar panels into one detector and the combination of the two sets into the other detector, which is sufficient to provide direction information.

\sysname uses federated energy storage~\cite{jhester:ufop:sensys} to power its microcontroller and peripherals.
Harvested solar energy is fed into a common first-stage storage capacitor and then automatically federated to its peripherals.
Federating energy allows us to prioritize detection and computation while saving up energy for more energy-expensive radio transmissions.
It also improves harvesting efficiency and allows separation of peripherals without fear that the microcontroller will lose power due to a radio transmission.
\sysname currently supports connections for two peripherals --- a Texas Instruments CC1101 radio and an extra slot for potential expansion to be used in future work.
%We plan to add additional sensors to our design in the near future.
\begin{figure*}[t]
	\centering
	\begin{subfigure}[b]{0.5\textwidth}
		\centering
		\includegraphics[width=\columnwidth]{figs/tracesin.pdf}
		\caption{Walking in.}
		\label{fig:tracesin}
	\end{subfigure}%
	%
	\begin{subfigure}[b]{0.5\textwidth}
		\centering
		\includegraphics[width=\columnwidth]{figs/tracesout.pdf}
		\caption{Walking out.}
		\label{fig:tracesout}
	\end{subfigure}
	\caption{These traces show example solar panel voltages and detector outputs over time when a person walks through a \sysname-enabled doorway. The top traces show how the solar panel's voltages are deformed during the doorway event. The detector triggers are used to wake up the microcontroller and detect events and their direction. The angling of the panels cause the inward facing and outward facing detectors to trigger at different times depending on the direction the person is walking.\label{fig:traces}}
\end{figure*}

\subsection{Detection}
\label{sec:detection}



%\fxnote{[It would be nice to do this data driven off some traces gathered. Simulate, essentially, just by doing some light math. So how do we capture the waveforms when we trigger the wakeup? - JDH]}

%what happens when a person walks through.
When someone walks under \sysname, they block some of the reflected light hitting the solar panels.
In \figref{fig:traces}, the ``solar'' traces on top shows how the voltage from the solar panels changes during a doorway event.

In order to detect a doorway event, we could use an ADC to continuously measure the solar panel voltage over time and analyze those readings to detect the presence and, more importantly, direction of motion.
Voltage levels and waveform shapes vary with lighting conditions, especially when one side of the doorway has more natural light.
This approach would require sophisticated signal analysis and prohibitive energy consumption.
Instead, \sysname uses a \textbf{detection circuit} that wakes up the microcontroller when it detects a significant change in the solar panel voltage over a short period of time.
This circuit consists of a passive first-order capacitive filter connected to a nano-power comparator---producing a square wave that transitions when the voltage increases or decreases faster than a set rate.
These transitions trigger interrupts that help \sysname detect when someone is passing through the doorway.

In order to determine movement direction, we use two detector circuits: one that detects change on the outward-facing panels and another that detects change on the combined inward- and outward-facing panels.
When someone walks through the doorway, the detectors trigger at different times, depending on the walking direction, as shown in \figref{fig:traces}.
\sysname compares the timing of these detector interrupts to distinguish incoming and outgoing doorway events.


\noindpar{Removing light flicker.}
Many fluorescent indoor lights flicker at \SI{60}{\hertz} or higher---a much higher frequency than the events \sysname is designed to detect.
These fluctuations can confuse the detection circuit and produce false positives unless they are filtered out.
We add a low-pass filter to remove noise above \SI{10}{\hertz} from the solar panel signal.
%\figref{fig:flicker} shows the impact of the noise to the light signal both before and after filtering out the higher frequencies from the signal.

\noindpar{Isolating harvesting from sensing.}
If connected directly, \sysname's harvesting and event detection circuits conflict in two important ways.
First, the harvesting circuit stores harvested energy in a \SI{100}{\micro\farad} capacitor---a size that ensures that \sysname can store enough energy for short-term tasks and dampens the low-frequency voltage fluctuations that we need in order to detect doorway events.
Second, short-term power spikes from interrupt service routines and other computation cause high-frequency dips in the solar voltage, which can confuse the detection circuits.
We address both of these challenges by adding an additional low-pass filter between the detection and harvesting circuits. This isolates the solar panel from the load, and allows the solar panel voltage (after the initial flicker filter) to fluctuate over a wider range in response to doorway events with less interference from the storage capacitor, the microcontroller power draw, and the detector circuit power draw.


\noindpar{Detection algorithm:}
A high-level overview of our detection algorithm is shown in \figref{fig:software}. During normal operation, when no doorway events are detected, \sysname's MCU remains in deep sleep.
While in deep sleep, the MCU is only triggered awake by the detector circuits going from high to low---designating the beginning of a doorway event, from the change in solar harvesting energy from the light occluded by a person walking through the doorway.
Once triggered, the MCU starts a timer (a few seconds), and records the time at which the interrupt occurred, then goes into sleep mode, waking up throughout the doorway event to capture the length of time between each detector's status change (from HIGH to LOW and vice versa).
Multiple interrupts often fire during a single doorway event as a person does not block light to the panels in an exact and smooth manner.
The timer defines the boundaries for what will be considered part of the event.

\begin{figure}[t]
\centering
\includegraphics[width=0.75\columnwidth]{figs/software.pdf}
\caption{The \sysname detection algorithm and system decision flowchart. The algorithm is composed of three parts that handle doorway events, maintenance, and idle waiting. \label{fig:software}}
\end{figure}


Times are recorded for the first falling edge interrupt and the last recorded rising edge for both solar panel groups.
When the timer fires, both solar panel groups' start and end times are compared to determine the direction of the event (entry or exit), and the \sysname stores the detected event in non-volatile memory.


In rare cases, only one detector detects the event.
These events are reported as a partial event, which doesn't have direction information.
Partial doorway events can occur when a person walks by the doorway, but not through it (just close enough to interfere with one panel group).


\subsection{Communication and Infrastructure}
The data that \sysname collects about people walking through the doorway is stored in non-volatile memory until the system has enough energy to make a radio transmission.
The current setup collects data for a certain fixed number of events before it polls the radio to see if it is available.
If there is enough available energy, the system will send statistics for the collected data, clearing its buffer.
If the capacitor for the radio isn't sufficiently charged, it will go back to sleep and try again after each subsequent event.
This way, we can keep collecting data as events occur and transmit it all to the base station when we have sufficient power to do so.
%\fxnote{[Remove this - JDH]}

%\subsection{Adaptation}
%\fxnote{[Remove this - JDH]}

%\sysname dynamically changes task sets depending on energy availability - attempting to assign higher energy tasks when energy is more available in the environment and assigning lower energy tasks assigned when energy is scarce in the environment. \fxnote{[Does not do this, need to remove - JDH]}
%These tasks correspond to different tiers of quality of service (QoS) of the application, with the highest level providing real time event marking on door traffic coupled with person identification using heights, hair color, and wardrobe. \fxnote{[Does not do this either, need to remove - JDH]}
%The lowest QoS tier corresponds to only logging entry and exit, and sending a summary of the data opportunistically over the radio. \fxnote{[Does not do this either! Put in future work- JDH]}

\section{Implementation}
\label{sec:implementation}

\begin{figure*}[t]
    \centering
    \begin{subfigure}[b]{0.5\textwidth}
        \centering
        \includegraphics[width=\columnwidth]{figs/panels_lossy.jpg}
        \caption{3D printed solar panel enclosure with angled slots for solar energy harvesters.}
        \label{fig:mounting}
    \end{subfigure}%
    %
    \begin{subfigure}[b]{0.5\textwidth}
        \centering
        \includegraphics[width=0.75\columnwidth]{figs/board.png}
        \caption{ \sysname prototype PCB.}
        \label{fig:pcb}
    \end{subfigure}
    \caption{\sysname implementation \label{fig:prototype}}
\end{figure*}


We have implemented a prototype \sysname sensor for evaluating our approach, including custom hardware in the form of a Printed Circuit Board (PCB) (shown in \figref{fig:pcb}), firmware for managing the doorway sensing application, and a custom 3D printed doorway mounting system that holds the assembled PCB and solar panels in a slim profile \figref{fig:mounting}.

\noindpar{Hardware:} Our prototype hardware integrates four (4) RL-55x70 solar panels (70.00mm x 55.00mm) from Seeed and a custom printed circuit board~(PCB) held together by a 3D-printed plastic enclosure, detailed later in this section.
The prototype's hardware is composed of an MSP430FR6989 microcontroller from Texas Instrument's (TI) FRAM line of ultra-low-power processors.
The newest FRAM-based MSP430s have several advantages over previous models: lower sleep-mode currents, shorter wake-up latencies, and faster non-volatile FRAM.
Using the faster wake-up capabilities, \sysname is driven entirely by interrupts and remains asleep most of the time to conserve energy when not in use.
The solar panels are connected in two banks, where both of these banks are made up of two panels connected in parallel.
Both these banks are connected in series with each other to increase the harvesting voltage, allowing for greater volatility in voltage which makes it easier to recognize features of the signal.
This configuration provides enough current to power the circuit with sufficient voltage levels for detection, as well as powering the system.
The detector circuitry is made using nano-power comparators (TI TLV3691) and a passive RC filter network. The RC filter network is tunable using trim potentiometers pre-installation, or digital potentiometers in deployment.
The \sysname PCB also has a TI CC1101 radio for communication.
The hardware used in the \sysname prototype, shown in \figref{fig:pcb}, is not prohibitively expensive or obtrusive.
% solar panels are expensive: $14 for 8, device cost is 19.37
The total cost of the current prototype, including all PCB, parts, assembly costs, and solar panels is \$22.63 per unit if ordered in quantities of 1000. The distribution of the prototype costs are showed in Table \ref{tab:costbreakdown}.
The current prototype has several components that are meant to enable experimentation and testing (modular board design, jumpers, headers, test points, etc) -- a commercial version of \sysname will be dramatically cheaper and smaller.
\begin{table*}[t]
\footnotesize
\caption{Detailed breakdown of the \sysname prototype}
\label{tab:costbreakdown}
\begin{tabular}{@{}p{1.4in}llc@{}}
\toprule
\textbf{Components}          & \multicolumn{1}{r}{\textbf{Cost per individual unit}} & \multicolumn{1}{r}{\textbf{Unit Cost for 1000 units}} \\ \midrule
\textit{Solar Panels}       	& \$ 7.8	&  \$ 7	 \\
\textit{Microcontroller (MSP430FR6989)} & \$ 8.3	& \$ 4.71	\\
\textit{Components} & \$ 18.26	& \$ 8.72     \\ 
\textit{PCBs} & \$ 19.44	& \$ 2.2     \\ \midrule
\textit{Entire Waldo Prototype} & \$ 53.8	& \$ 22.63    \\ \midrule
\end{tabular}
\end{table*}

\noindpar{Firmware:}
The \sysname firmware implements the detection algorithm discussed in \secref{sec:system}.
Monitoring the interrupts from the detectors and deducing the direction of motion upon triggering are the main tasks of the system.
The firmware is designed to be ultra-low power, even in active mode, and has low computational complexity, offloading the bulk of the detection to the hardware circuits.
The \sysname firmware is composed of 398 lines of commented C code, compiling to a 2110 byte image. This code size comprises only 1.6\% of the available code space on the MSP430FR6989 (128KB), leaving ample room for implementing custom tasks, recognizers, or multiprogramming operating systems.

\noindpar{Mechanical Design:}
The 3D printed mounting system (shown in \figref{fig:mounting}) is made of PLA plastics and contains the PCB, solar cells, and necessary wiring connecting them.
\sysname's 3D printed enclosure measures \SI{13.2}{\centi\meter} by \SI{47.0}{\centi\meter} by \SI{1.0}{\centi\meter} at its thickest point. The enclosure provides a nesting place for the solar cells, pointing downward.
The angle of the solar cell slots is set such that some solar cells tend toward the entry, while the rest toward the exit.

All software, firmware, hardware schematics and layouts, and 3D printed mounting system will be made freely available at publication time.

\section{Evaluation}
\label{sec:evaluation}
In order to evaluate the efficacy of our approach, we evaluated \sysname in both controlled and in-the-wild deployments.\footnote{Both controlled and in-the-wild deployments were approved by our Institutional Review Board.}
We also compared \sysname to a similar commercial sensor during the in-the-wild deployments, which is discussed in greater detail in Section \ref{sec:enocean}.   

\subsection{Controlled Experiments} 
We evaluated \sysname's performance under controlled conditions in three phases to test different variables the system might encounter:

In \textbf{phase one} we tested \sysname on multiple doorways, with different light levels, flooring, doorway heights, and doorway widths.
For each test, we evaluated \sysname's ability to detect someone passing through and determine the person's direction of movement.
We also tested the system's robustness to variations in height, clothing, and hair color by including a diverse group of subjects.
We tested on \numDoors different doorways with \numPeople people for a total of \numExp different doorway events (each person walked through multiple times per doorway).

In \textbf{phase two} we tested the limits of the device, examining the factors that affect its accuracy, performance, and availability---including lighting conditions, walking speed, and short delays between doorway events.
We also tested a variety of events that may be falsely detected as doorway events.

In \textbf{phase three} we explored the energy-harvesting ability and gather microbenchmarks of the energy consumption of the parts of the \sysname system.

\begin{table*}[t]
	\footnotesize
		\begin{tabular}{@{}p{1.0in}p{0.04in}ccp{0.04in}p{0.04in}ccp{0.04in}p{0.04in}ccp{0.04in}p{0.04in}cc@{}}
		\toprule
		\multirow{2}{*}{\textbf{Passageway \#}} & \multicolumn{4}{c}{\textbf{Light Intensity (lux)}} & \multicolumn{4}{c}{\textbf{Flooring}} & \multicolumn{4}{c}{\textbf{Dimensions (cm)}} & & \textbf{Total} & \textbf{Direction} 		\\
		          & & Inside & Outside & & & Inside & Outside & & & Height & Width & & & \textbf{Events \#} & \textbf{Accuracy(\%)}\\\midrule
		Doorway 1 & & 98\textsuperscript{*} & 93 & & & Tile   & Tile & & & 202 & 88  & & & 119 & 94.1    \\  %Lab Door
		Doorway 2 & & 81\textsuperscript{*} & 74 & & & Carpet & Tile & & & 203 & 88  & & & 128 & 100.0   \\  %Jacob's office
		Doorway 3 & & 60                    & 57 & & & Carpet & Tile & & & 203 & 88  & & & 123 & 93.5    \\  %Tutor Room
		Hallway 4 & & 71                    & 69 & & & Tile   & Tile & & & 236 & 243 & & & 123 & 97.6    \\  %HW0 - Lab
		Hallway 5 & & 59                    & 59 & & & Tile   & Tile & & & 236 & 240 & & & 127 & 99.2    \\  %HW1 - Office
		Hallway 6 & & 56                    & 62 & & & Tile   & Tile & & & 236 & 244 & & & 118 & 89.8    \\  %HW2 - Tutor
		Hallway 7 & & 82                    & 71 & & & Tile   & Tile & & & 221 & 191  & & & 143 & 99.3    \\  %Bathroom Hallway
		\bottomrule
		\end{tabular}
		\caption{Evaluation results with \numPeople test subjects having variable height, hair color, and clothing as described in \secref{sec:normal_operation}. We tested \numDoors different doorways/hallways of varying light levels, dimensions, and flooring types, all of which had enough light to power \sysname. We ran multiple people through each of these \numDoors passageways one at a time, noting the detection accuracy and how many of the detected events had correct direction. All controlled events were detected so we display the direction accuracy of those events above.  All these results show that an adequately lit \sysname occupancy sensor can accurately detect doorway events and their directions.
		\vspace{1mm}
		\\\textsuperscript{*}Mixed Lighting --- Combined natural and artificial light
		\label{tab:detection}}
	
	\end{table*}	

\subsubsection{Methodology and Claims}
The following experiments address the goals defined in \secref{sec:system}.
We address system availability~(Goal~1) by demonstrating the low power draw of the system and the number of recorded doorway events (and the number of doorway events missed) for each doorway test.
% the minimum light intensity at which \sysname can sustain operation and detect doorway events.
Further, we evaluated the accuracy in determining the direction~(Goal~2) by observing how often \sysname correctly determined walking direction.
We explored variable lighting conditions~(Goal~3) by testing the device under \numDoors different doorways and hallways with diverse lighting conditions, both typical and adverse.
We address human variation~(Goal~4) by evaluating different walking speeds and the effects of clothing and hair color/hair covering on detection patterns.
We claim that form factor~(Goal~5) is addressed by our prototype and slim mechanical design, described in \secref{sec:implementation}.

We also tested the limits of the device, by varying different factors to see when the device stops working and exploring conditions that can confound the sensor.
These tests cannot hope to cover all possible deployment conditions, but they do give a broad sense of the capabilities and limitations of \sysname.

We gathered all electrical signal measurements, except where specified otherwise, using the Saleae Logic~16 logic analyzer\footnote{https://www.saleae.com} at a sampling rate of 5KS/s.
The analyzer's high-impedance ADCs allow for unobtrusive signal monitoring.
This sampling rate is sufficient to detect the types of slow-varying doorway events that human activities produce.
We manually recorded the direction of each doorway event as ground truth to verify \sysname's event detection accuracy, then compared the ground truth results with the results measured by the logic analyzer.
We measured light intensity levels using a TSL2561 light sensor,\footnote{https://cdn-shop.adafruit.com/datasheets/TSL2561.pdf}
aligned to the same angle as the solar panels in both directions to get accurate light intensities falling on the panels.

Finally, we investigate the accuracy of \sysname against our manually gathered ground truth (visually verifying a person entering or exiting the room) instead of comparing to another occupancy-detection system.  
We do compare Waldo to a commercially available sensor in the later discussion on uncontrolled deployment.

%%only if time allows:
%%%passing vs. normal entering and exiting of the room
%%%what multiple people entering in close succession looks like to the system and can the system tell the difference between the systems.

%Notes on current testing plan environment:
%tile floor, semi reflective
%door open
%lights on on both sides of the door
%16 solar panels that are alternatively tilted by 10 degrees in either direction (8 panels on each side)
%walking at normal speed through the center of the doorway

%In order to test how well \sysname fares in terms of the availability goal, we tried turning down the light levels till \sysname stopped sustaining operation. Our aim in performing this experiment is to find a threshold above which \sysname sustains operation and is available for detecting ephemeral doorway events as they occur.
%We found that \sysname stops sustaining operation below XXYYZZ lux. This is an acceptable threshold as the average light levels in office environments will be greater than or equal to \SI{70}{\lux}. Even when the light levels fall significantly lower till \SI{40}{\lux}, we are able to detect events but it does affect our direction detection, as outlined in the next section.


\subsubsection{Normal Operation}
\label{sec:normal_operation}

In order to evaluate how well our approach detects doorway events, we tested \sysname across multiple different doorways with a diverse group of subjects.
In these tests, we focused on detecting doorway events caused by a person walking \textit{under} the doorway and accurately determining the direction of the person's movement.

\noindpar{Experiment Overview:}
We tested \numPeople different participants, with different physical characteristics---heights ranging from 5'4'' to 6'4'' and hair colors including blond, brown, black, and bald.  Our test group included a wide range of clothing colors (light and dark) and a variety of head coverings (hats, beanies, and hijabs).


For this experiment, we attached \sysname prototypes to the top of \numDoors different doorways and hallways.
\tabref{tab:detection} describes the passageways, including light intensity levels, flooring type, and dimensions.
%For each, we maintained test subject diversity, in order to characterize \sysname's performance, independent of the characteristics of individual subjects.
For doorways with doors, the door remained open throughout the experiments.
%This was done since
Due to differences in subject availability, we had seven of the participants walk into and out of the room or hallway at least 10 times in each direction on all \numDoors passageways.  An additional two participants were asked to walk in and out of the doorways and Hallway 7 at least 5 times in each direction.  When participants were able to complete more than the requested 10 passes, that additional data was collected as well.  Some additional data was generated for these experiments since passersby would occasionally trigger the system.  For the controlled data collection, we discarded events detected by the system when they were affected by someone other than the intended subject triggering the system, like a person passing by.

\noindpar{Results:}
The results of the controlled experiment, including \numExp individual doorway events, are shown in \tabref{tab:detection}.
Each \textit{event} consists of one person walking through one doorway one time.  Participants walked through many different sides of the doorway, not just through the center each time, and they varied their entry and exit paths throughout the runs.
Participants also choose their walking speed at each run; while most chose a natural walking pace, some did vary their speed occassionally. 
\sysname successfully detected \SysAccuracy of the \numExp doorway events.
\sysname also determined the walking direction correctly for \numDir (\dirAccuracy) of the events.
\sysname's performance was consistent across all test subjects, independent of human variations like height, gait, hair color, and clothing.



\subsubsection{Factors affecting \sysname's operation}
\label{sec:confounding}

%\noindpar{Experiment Goals:}
In addition to testing ``normal'' walk-through conditions, in this section we examine factors that affect \sysname's performance as an occupancy-monitoring sensor.
It would be impossible to exhaustively study all possible variations, but we are able to explore how \sysname reacts to a variety of conditions and behaviors that it will encounter in actual deployments.
Specifically, we explored the following factors:


\paragraph{Light intensity:}
We tested \sysname on a variety of doorways with varying lighting conditions, with results listed in \tabref{tab:detection}. 
Since \sysname's solar panels are sensitive to visible light and the IR spectrum, we used a TSL2561 sensor to measure both mixed signal (visible and IR) data along with purely IR data, and recorded the combined illumination value (in lux).
Our current prototype is fully functional on doorways with light levels above 56~lux on both sides.
An average room/hallway in an office-style setting has light levels around 70~lux, which is sufficient to power the \sysname sensor.
It is worth noting that we can customize \sysname for exceptionally dark doorways either by increasing the number of solar panels without changing the working of the system itself, or by employing input booster circuits like the ones used in CleanCut~\cite{colin2018cleancut}.

\paragraph{Walking Speed:}
\sysname detects people walking under doorways based on the changes they induce in the system's harvested energy supply.
This means that if a person walks slowly enough, their movement should become imperceptible to the system.
In order to evaluate this limit, we asked test subjects to walk under the sensor at different speeds.
We used a metronome to which the subjects could match their steps in order to achieve a consistent, even speed.
With extremely slow walking (slower than 1~ft/s), we did observe decreased accuracies.
\sysname occasionally detected a slow-moving doorway event as two events.
No test subjects have yet been able to walk slowly enough to avoid detection entirely.
We don't consider this to be a problem for \sysname, since in practice, people don't often move at such slow speeds.

\paragraph{Door Width and Height:}
All doors (in \tabref{tab:detection}) were around 203 cm tall by 88 cm wide.  All hallways tested were 221 to 236 cm tall and 191 to 244 cm wide.
%A typical interior doorway is 32 inches by 80 inches.
In our experiments, the door width and height had no significant effect on the accuracy; however, the controlled experiments only tested when a single person went through a wide door at a time, and we did not control for participants walking through the middle or side of the door (they were asked to walk naturally) or their entry and exit angles around the passageway location.


\begin{figure*}[t]
    \centering
    \begin{subfigure}[t]{0.35\textwidth}
        \centering
        \includegraphics[width=\columnwidth]{figs/spotlight.pdf}
        \caption{These traces show the solar panel output in the presence of the ``Spotlight'' effect. The top figure shows the response when someone walks \textit{across} the ``Spotlight'', while the bottom one shows the response when someone walks \textit{through} the door.}
        \label{fig:spotlight}
    \end{subfigure}%
    %
    \hspace{30pt}
    \begin{subfigure}[t]{0.35\textwidth}
        \centering
        \includegraphics[width=\columnwidth]{figs/acrossAndLinger.pdf}
        \caption{ This figure compares a person walking \textit{through} the doorway (top two traces) versus walking \textit{across or by} the doorway on the outside. There is a clear delay between the two solar panel channels when someone walks through, whereas the change is reflected simultaneously when someone walks by.}
        \label{fig:across}
    \end{subfigure}
    \caption{Factors affecting \sysname operation. \label{fig:confounds}}
\end{figure*}


\paragraph{Multiple people:}
\secref{sec:normal_operation} showed the ability of \sysname to detect individual people walking through.
A practical consideration would be to examine the performance of \sysname when multiple people walk through.

In order to evaluate this, we tested two subjects walking through doorway \#1 with varying time delays between them.
This gave us control over the time separation between two events, and allowed us to examine how closely can two people walk in without being detected as one, quite large person.
We discovered that as long as two people have at least \minSeparation between them, \sysname can accurately distinguish between them, but may incorrectly classify the direction.
This limitation is introduced due to the time required by the solar panels to reset or stabilize before the next event can occur.
A subsequent logical conclusion is that if two people walk side-by-side, \ie with zero separation between them, out current prototype is unable to detect them as two events.

\paragraph{The ``Spotlight'' effect:}
An interesting consequence of light-based detection is a problematic condition that can occur especially in mixed-light settings, when an intense low-angle light source dominates the illumination.
This effect appears in the presence of a very focused source of light that dominates the illumination around the doorway, such as a spotlight or a west-facing window in late evening when the sun blazes directly through.
When someone walks across the light source, even if they are far from the doorway, it can be detected falsely by \sysname as someone walking through.
\sysname detects people based on a decrease in the harvested energy and momentarily blocking the spotlight can produce a significant decrease in voltage on both solar channels.
Interestingly, we can see from \figref{fig:spotlight} that the raw output of the solar panels look sufficiently different for someone walking \textit{across} the focused source as compared to when someone walks \textit{through} the doorway in presence of a focused source.
%At present, \sysname is equipped to detect events with good accuracy.
With further signal processing, \sysname could distinguish these spotlight events so that such events would not cause false triggers.


\paragraph{Detection Range/Walking across, not through:}
\label{subsubsec:range}
Considering that \sysname uses the blocking of light to detect a person, there will be an influence radius inside which a person starts affecting the sensor.
If someone walks by either side of a doorway monitored by the \sysname sensor and are within the radius, they will trigger the detector circuits and register as an event by \sysname.
We ran an experiment to determine this radius of influence where the subject was directed to walk by on either side of the doorway at increasing distances from the sensor.
We started with a distance of 30 cm (~1 foot) and went up to 152 cm (~5 feet), in increments of 30 cm (~1 foot).
For each distance, we asked the subject to walk by multiple times and recorded how many false triggers were detected.
An example of this is shown in \figref{fig:across}.
We have observed that under typically indoor lighting conditions for distances greater than 91 cm away from the doorway, there is a negligible chance of triggering false events.

It is interesting to note from \figref{fig:across} that there is a distinguishable difference between this event as compared to someone walking through the doorway.
Since they are walking only on one side of the doorway, their effect on both channels is not delayed by the angling of the solar panels, as is the case with walking through.
As with the ``Spotlight'' effect, we should be able to extract this difference with further signal processing and learning. %and we attempted a stab at addressing this confounding condition further in the uncontrolled experiments that will be discussed later.  %\fxnote{[This might need some rewording for the fact that we do try to handle this in the uncontrolled experiments.-NT]}

\paragraph{Lingering in the doorway:}
Another situation that causes false triggers is when a person approaches the doorway, but simply pokes their head in.
Upon evaluation, we discovered that as long as the person is poking their head in the doorway, the solar panel output remains at a lower level, and when they exit, it rises back again.
Although the current system implementation isn't equipped to differentiate between someone passing through and someone lingering in doorway, there is a clear difference in the raw waveform outputted by the solar panel.
This case is similar to \secref{subsubsec:range} in terms of being distinguishable from a person walking through and with some careful, direct signal processing it is definitely possible to differentiate between the actual and the confounding case.

\subsubsection{Microbenchmarks}
\label{sec:microbenchmarks}
The more effective \sysname is at maintaining a low-power state when idling, the more available \sysname is for detecting doorway events and monitoring occupancy.
The energy requirements for detection and active computation must be kept low as well.
Unlike intermittent computing systems, \sysname must intentionally avoid power failures.
We measured the current draw of our \sysname prototype while it was mounted on doorway \#1.
The idle draw of the system was \SI{7}{\micro\ampere}, showing that \sysname can survive in a doorway with minimal light and energy harvesting.
We gathered other benchmarks of system energy performance in each of sysname's different operating modes. To seperate harvesting and consumption, these measurements were made after the MIC841 hysteresis chip. So, the actual power and energy is slightly higher (by \SI{1.5}{\micro\ampere} according to the datasheet).


% Idle Current: 7-11uA, MCU not active
% Timer / ISR handling, single detector, no event : 190-230us, MCU active
% Doorway Event, MCU active, compute time: 7.2ms
%
% Peak current for active: 500uA
% Avg current for active: 220uA
% 2.8V
\begin{table*}[t]
\footnotesize
\begin{tabular}{@{}p{1.4in}llc@{}}
\toprule
\textbf{State}          & \multicolumn{1}{r}{\textbf{Avg. Current}} & \multicolumn{1}{r}{\textbf{Peak Current}} & \multicolumn{1}{r}{\textbf{MCU Active}} \\ \midrule
\textit{Waiting (Sleep Mode)}       	& \SI{7}{\micro\ampere}	&  \SI{11}{\micro\ampere}	& \textcolor{magenta}{\xmark} \\
\textit{Maintenance Actions} & \SI{200}{\micro\ampere}	& \SI{250}{\micro\ampere}		 & \textcolor{green}{\cmark} \\
\textit{Doorway Event Handler} & \SI{500}{\micro\ampere}	& \SI{700}{\micro\ampere}	    & \textcolor{green}{\cmark} \\ \midrule
\end{tabular}
\caption{Microbenchmarks for \sysname current consumption.}
\label{tab:microbenchmarks}
\end{table*}


Since \sysname is event-driven, its actual power consumption varies depending on the activity underneath the sensor. As shown in \tabref{tab:microbenchmarks} the idle power draw of \sysname is low (\SI{7}{\micro\ampere}). When timers or detector circuits trigger interrupts (maintenance events) the system draws \SI{440}{\micro\watt} for a few \si{\micro\second}.
Computing walking direction, storing data, and transmitting results when an event ends is more expensive~(\SI{1100}{\micro\watt}, on average).
During typical operation, these higher-power events account for an insignificant fraction of the device's runtime, and the average power draw is often indistinguishable from the idle draw.
Overall the energy consumption of the system is low, but could be further improved with careful tuning of resistance values, sleep states, and the analog circuitry.







\subsection{\sysnames in the Wild}


\begin{figure}[t]
\centering
\includegraphics[width=0.8\columnwidth]{figs/floorplan.pdf}
\caption{\sysname deployment locations for the in-the-wild experiments. Each location features different lighting conditions as well as traffic patterns resulting from the adjoining labs, offices, and classrooms. \label{fig:floorplan}}
\end{figure}

%setup

In order to understand how \sysname behaves in uncontrolled conditions, we deployed multiple \sysname units along a hallway that connects offices, labs, classrooms, and bathrooms at locations shown in \figref{fig:floorplan} for a collective total of \ITWdays.  


\subsubsection{Experimental Setup.}

\begin{table*}[t]
\footnotesize
	%\begin{tabular}{@{}p{0.65in}>{\centering\arraybackslash}p{0.4in}>{\centering\arraybackslash}p{0.4in}p{0.4in}p{0.4in}p{1.1in}p{0.55in}p{0.55in}p{0.8in}p{0.55in}@{}}
	\begin{tabular}{@{}p{1.0in}p{0.2in}llp{0.2in}p{0.6in}p{1.3in}p{0.6in}l@{}}
	\toprule
	\multirow{2}{*}{\textbf{Location}}	&	\multicolumn{4}{c}{\textbf{Dimensions (cm)}} & \multirow{2}{*}{\textbf{Lighting}} & \multirow{2}{*}{\textbf{Traffic Profile}} & \textbf{Days}& \textbf{Ground Truth}  \\
		& & Height & Width 	&  						&	&  &   \textbf{Deployed} & \textbf{Events} \\\midrule
	W1 & & 243 & 235  & & Indoor  & Light/Moderate   & 17    & 436   \\ %Hallway1
	W2 & & 221 & 180 & & Indoor  & Light/Moderate    & 18   & 923  \\ %Hallway2
	W3 & & 202 & 149 & & Mixed   & Moderate/High/Bursts& 10 & 1067  \\	%Stairs
    	W4 & & 236 & 241  & & Indoor  & Moderate/High/Bursts & 10& 1070 \\ %LabHW
	W5 & & 239 & 190 & & Indoor  & Moderate/High/Bursts  & 9& 1294  \\ %preHW2
	\bottomrule
	\end{tabular}
	\caption{In-the-Wild deployment location descriptions and counts of ground truth events over the days deployed.
	\vspace{1mm}
	\label{tab:ITWcategoryfreq}}

\end{table*}

We conducted in-the-wild experiments in two sessions. In the first, sensors were deployed at two locations (W1 \& W2) for 24 days at the end of an academic semester and into the holiday class break.
We observed events on 18 of the days (only 17 for one of the sensors). 
In the second session, we deployed at three different locations (W3--W5) for an additional 11 days, with events recorded on only 9--10 days. This second session, at the start of a new semester, had heavier traffic, as shown in \tabref{tab:ITWcategoryfreq}.
The locations differed in light levels, width, and height, while providing a range of lighting and behavioral conditions. 
For example, the sensors near a classroom encounter multiple confounding cases like lingering and crowds passing under a doorway, while the ones near a lab or office are affected by lingering, spotlights, and occasionally crowds.
We selected hallways in order to observe a wider range of natural traffic patterns, including multiple people walking together under the sensors. 
At each location, we installed a \sysname sensor, a commercial ceiling-mounted EnOcean occupancy sensor~\cite{EnOcean}, and a camera to provide a ground truth confirmation of hallway activities. %Dimensions for these locations are reported in \tabref{tab:ITWresults}.  %The lab doorway is 34.5" x 79.5" and the light level is 86 lux inside the room and 91 lux outside of the room. The classroom doorway is 35" x 87" and the light level is 91 lux inside the room and 100 lux out of the room. The hallway is 76" x 93" and the light level on one side of \sysname is 94 lux and 89 lux on the other side. 
All locations have tile flooring on both sides of the sensors and are typically well enough lit to transmit a packet once after at least 30 seconds have passed.  This is usually ample time for the system to send a packet for every five detected doorway events or heartbeat (nothing has happened in two-minute) events, as events take at least 6 seconds each to process. 
%\hey{I find it a bit awkward to define this in terms of number of doorway events. Since event rate can very wildly. Can we instead relate this to time?} 
We also deployed wall-powered basestations to collect the transmitted data from the batteryless \sysname devices and EnOcean sensors.
Each basestation is an Internet-connected Raspberry Pi with a CC1101 radio and an EnOcean receiver that receives packets and stores them in an SQL database for later retrieval. 
We deployed two base stations for redundancy---one would have been sufficient.   

At each instrumented passageway, we also place a video camera that continuously collects ground truth information by recording the actual doorway events. 
We manually labeled all recorded events by watching the videos and annotating by storing the time and a description of each event---\textit{in, out, pass-by}, as well as more complex cases like lingering, people changing direction under the sensor (u-turns) and multiple people passing in or out in a group. 
In order to make sense of the wide range of observed behaviors, we sorted the events into 3 different categories: simple events, multiple-people events, and complex events. 
Simple events include simple ins, outs, and very close pass-by events with only one person around the sensor within a 6-second window of time.
Multiple people events involve multiple people that pass under the sensor going in the same direction within a 6-second window.  
These events range from 2 people walking side-by-side or close succession to 23 people all exiting at the same time when classes let out.
All other events fall in the complex category, including lingering, changing directions underneath the sensor, and multiple people going in different directions under the sensor.  

We compare these ground truth events against the sequence of events \sysname detects.
\sysname send a message to the basestation once it records at least five events or heartbeats and has enough harvested energy.
\sysname generates heartbeat events after two minutes of inactivity. 
So, during long periods of inactivity, \sysname will generate and send a packet every 10 minutes (consisting of 5 heartbeats) to let us know that it is still alive and has not seen any new events. 
This heartbeat frequency was selected for this experiment to help us distinguish between periods of inactivity (no detected events), dropped packets, and system failures.
This frequency can also be adjusted to balance these liveness concerns with energy budget constraints.

\begin{figure}[t]
\centering
\includegraphics[width=0.9\columnwidth]{figs/timeline.pdf}
\caption{ The ground truth event timeline of our in-the-wild deployment and how it is divided into packet windows based on the when the packet is recieved immediately after processing the fifth event record within the packet window. The heart at event 4 represents a heartbeat event generated by \sysname when there have been no detected events for 2 minutes.  The two examples of recorded event and missed ground truth events represent where two individuals walk through in quick succession, such that the sensor only identifies them as one event.  This example provides a high-level view of how we compute the event statistics within a window.  \label{fig:eventtimeline}}
\end{figure}

Each network packet that \sysname sends includes the sequence of the events and heartbeats, an estimate of time that has past between each event, their classifications, and a CRC computed over the recorded values to protect against packet corruption. 
As a simple redundancy mechanism in case of packet loss, each radio transmission includes the last two previously-sent packets along with the current packet. 

For simplicity, \sysname currently has no sense of absolute time---just relative inter-event time.
In order to reconstruct the sequence of events, the time the basestation received the packet is used as a reference to match windows of events to the ground truth data.
This time is closely correlated to the time of the packet's last event.
So, we use these times to match ground truth events with the events recorded by the \sysname sensors.

%Looking at high level detection of events
\subsubsection{Data Collection and Analysis Method.}
Each packet encodes the sequence of 5 events or heartbeats that \sysname recorded and an estimate of time that has past since the last event or heartbeat recorded.
A packet received at the basestation is matched with a series of ground truth events based on the packet's receive time, the estimated inter-event times, and the duration of \sysname's event window (\SI{6}{\second}).  
This helps account for minor human errors in labeling event times, as it is not always clear exactly when someone started affecting the sensor from video data.
If two ground truth events match a packet, the one closest (in time) is chosen and mapped to the 5th event associated with this packet.
Using the information stored in the packets about number of events and their estimated associated time between the events, we can estimate the match of the other ground truth events with their likely corresponding records in the packet record sequence to analyze hits or misses and if the classified direction was correct.  
\figref{fig:eventtimeline} shows a high-level example of the how ground truth events are divided by the packet window based on the received time of the packet in order to calculate correctly detected events and possible missed events, as well as when detected, if they were correctly classified. 
Each \sysname packet has a monotonically increasing packet ID, which we use to detect packet losses.
If a packet is dropped or corrupted, it can be reclaimed from the following transmission, which includes the last two packets sent. 
%We tossed out activity that was only seen on the very edge camera's video (like a shadow from activity out of camera view) but was likely out of the range of our sensor. 
	
%Packets that map to a ground truth event are considered \textit{confirmed packets}.
%All other packets that could not be mapped to an event (within 30s), are still valid packets that represent five events that \sysname detected and must be considered. 
%We insert these packets into the event sequence and mark them as \textit{false positive} packets.
%The false positive packets may still contain valid data associated with the ground truth.
%The fifth event that generated the packet may have just been missed in the ground truth. 
%These confirmed and false positive markers divide up the ground truth data into probable event windows in order to make a meaningful comparison with the packet summaries.
%We then gather statistics on the number of probable events associated with each packet. 
%Confirmed packets are included in the associated window counts, however, false positive packets, while still providing a meaningful anchor for the probable event window, did not have an event closely associated enough to add to the probable event window counts.  

%We reason about the number of events that \sysname actually detects in these ground-truth based probable event windows by drawing on two characteristics of the system design. 
%One, each packet encodes five events, and two, every packet will have a different packet id number, one greater than the last packet.
%Using this knowledge, we can figure out which packets, if any, we missed (what packet id numbers were absent on the basestation?) and how well \sysname \textit{detected events}, even if it may have classified events incorrectly.
%When all the ground-truth events in the event window were explained by the number of events in the anchor packet and missed packets, we considered the window to be accounted for. 
%When they are not, we know that either events were recorded (on video) but not detected by \sysname, denoted as \emph{misses}, or \sysname detected more events than were reported in the ground-truth data, which we called extras. 
%\figref{fig:eventtimeline} shows a high level example of the how ground truth events are divided by the anchor packet time in order to calculate correctly identified event, incorrectly identified events and misses. 
%These values are aggregated over all packet windows in the ground truth data. \hey{I don't understand what this paragraph is saying. Remove? update?}


\begin{table*}[t]
	\footnotesize
		\begin{tabular}{@{}p{0.9in}p{0.8in}lllp{0.12in}p{0.2in}llp{0.2in}ccc@{}}
		%\begin{tabular}{@{}p{0.45in}>{\centering\arraybackslash}p{0.6in}>{\centering\arraybackslash}p{1in}p{0.4in}p{0.6in}p{0.35in}p{0.35in}p{0.25in}p{}p{}p{}@{}}
		\toprule
		\multirow{2}{*}{\textbf{Location}} & \textbf{Ground Truth} & \multicolumn{3}{c}{\textbf{Frequency of Events by Type}} & &\multicolumn{4}{c}{\textbf{Simple Events}} & \multicolumn{3}{c}{\textbf{Accuracy}} \\
	 & \textbf{Events} & Simple & Multi-Person*   & Complex   & &  & In  & Out  &  & In        & Out      & Total        \\  
	\midrule
	W1   & 436   & 91\%   & 4\%    & 5\%  & & & 201   & 190  &  & 71\%   & 96\%    & 83.12\%                   \\ %Hallway1. 91 is really 91.5 so 92 but then they dont add to 100
	W2   & 923   & 89\%   & 4\%    & 7\%  & & & 426   & 388  &  & 99\%   & 77\%    & 88.82\%                   \\ %Hallway2
	W3   & 1067  & 83\%   & 14\%   & 3\%  & & & 505   & 291  &  & 91\%   & 88\%    & 90.32\%                   \\ %Stairs
	W4   & 1070  & 83\%   & 14\%   & 3\%  & & & 343   & 503  &  & 92\%   & 98\%    & 95.51\%                   \\ %LabHW
	W5   & 1294  & 85\%   & 10\%   & 5\%  & & & 579   & 512  &  & 97\%   & 93\%    & 94.96\%                   \\ %preHW2
		\bottomrule
		\end{tabular}
		\caption{Waldo in-the-wild deployment frequency of events by type at each deployment location and accuracy on how the system performed on classifying the simple ins and outs that were encountered over the deployment.
		\vspace{1mm}
		\\\textsuperscript{*}Multi-Person Events --- This category of events represents only multiple persons traveling under the sensor going in the same direction.
		\label{tab:ITWresults}}
	
	\end{table*}
	


\subsubsection{Results.}  
%break down by category
\sysname performed well when detecting activity that was taking place under the sensor and, when the activity was close enough, around the sensor.
All events that happened under the sensor were detected, and throughout the deployment, we lost only one packet due to transmission errors. 
Based on the previous and subsequent packet numbers, we know the sensor recorded something but we can not verify whether the one ground truth event that occurred during this time was detected correctly.
\tabref{tab:ITWresults} shows the frequency of the different event categories that the sensors experienced while deployed and \sysname's accuracy on classifying the simple in and out events where a single user walked under the sensor.

\textbf{Simple Events} significantly outnumbered the other two event categories across all sensor locations.
\sysname is specifically designed to detect simple one-person ins and outs, and these are the most common form of traffic we observed.
\sysname detected all of the simple events, and correctly classified their direction 83--95.5\% of the time, depending on the sensor location.
Nearly all misclassified simple events were misclassified as pass-by events, though a few out events were misclassified as in events. 

As mentioned before, only one of the deployment locations~(W4) was used for both gathering training data and our in-the-wild deployment.
While this location (unsurprisingly) outperformed the others, the others were close behind---indicating both that the trained model was able to work well when used in different lighting conditions and that future \sysname iterations might achieve small performance improvements by adapting the model in situ based on observed light conditions. 


\sysname also detected the \textbf{Multi-Person} and \textbf{Complex} events, including lingers, u-turns, and multiple people affecting the sensor in quick succession; however, the sensor was not always able to accurately estimate number of people passing by or their direction.

\textbf{Multi-Person Events}---multiple people pass together or in quick succession under the sensor in the same direction---typically result in undercounting. 
When the events completed within \sysname's \SI{6}{\second} event window---common when two people were walking side-by-side---the events were reported as an in event or an out event, and the event directions were nearly always classified accurately (comparable to the accuracies reported for the simple events).
%
When these events lasted longer than \SI{6}{\second}, \sysname reported a group of multiple consecutive events, with the first event usually classifying the event direction correctly and subsequent events often misclassified when \sysname's event windows often captured partial events.

\textbf{Complex Events}, including lingers, u-turns, near pass-bys, and multiple people passing the sensor simultaneously in opposite directions behaved similarly, producing a group of one or more consecutive events, except that the direction estimate for the first event in the group is also often incorrect.
Another key difference is that some complex events can result in overcounting.
For example, a single person lingering under a \sysname sensor for a few minutes will produce multiple consecutive events.
The longer the person spent underneath the sensor, the more of these events \sysname would record.


Both Multi-Person and Complex events represent confounding cases for \sysname---and challenging cases for technologies for monitoring human movement through buildings.
While we plan to address them in future improvements (\secref{sec:discussion}), for now their impact depends on traffic conditions.
Under usual passageway conditions, a user wanting to count people can treat isolated events (events with more than a \SI{6}{\second} gap between them) as single person events with reliable direction estimates and end up with slight underestimates.
Sequences of consecutive events (with no gap) can, for now, be treated as reliable activity measurements but not accurate people counts or direction estimates.


The first phase of our deployment (W1 \& W2), during end-of-semester traffic conditions, saw fewer groups moving together and less overall traffic, and 89--91\% of the observed events were simple in and out events with 9--11\% confounding events (mostly two-person side-by-side events and some lingers).
As traffic increased at the start of the following semester, locations W3--W5 saw an increase in overall traffic and confounding events increased to 15--17\% of the total events.
In spite of the traffic increase, both phases were dominated by simple events, and \sysname provided information suitable for accurate people counting.
Of course, in some extremely high traffic areas (e.g., the entrance to a sporting event or concert) we expect that \sysname would have a high number of confounding events and behave like an activity sensor providing less information about individual people and their direction.  


\subsubsection{Commercial Sensor Comparison}
\label{sec:enocean}
%We note that it is difficult to fairly compare the performance of different occupancy-monitoring systems except in their accuracy. For example, CeilingSee~\cite{yang2017ceilingsee} uses 16 devices to instrument a room, while \sysname and SonicDoor~\cite{sonicdoor-buildsys2017} place one device in the doorway, and AURES~\cite{shih2016aures} places a single device in the middle of a room. 
%For this reason, we investigate the accuracy of \sysname against our manually gathered ground truth (visually verifying a person entering or exiting the room) instead of comparing to another occupancy-detection system.
As mentioned earlier, in order to compare \sysname to its closest commercially available alternative, we deployed a batteryless commercial ceiling-mounted PIR occupancy sensor by EnOcean~\cite{EnOcean} alongside our \sysname sensors on each passageway. While other similar sensors are available, we selected the EnOcean sensor because it was actually a battery-free commercial option that did not use rechargeable batteries and had transparent product specifications easily available online.  This sensor also was more programmable for expermental repeatability and, at the time of purchase, more easily available in our country.
This sensor is the powered by harvested energy, and uses ambient light~(IR) changes to detect movement.
Unlike \sysname, this sensor only detects activity/occupancy (but no direction information).


EnOcean sensors send two types of packets: an \emph{occupied packet} when an event is detected and an \emph{unoccupied packet} after 10 minutes of inactivity has passed followed by every 30~minutes after that.  
If movement is detected, it sends an occupied packet to a receiver attached to the same basestation we use to receive packets from \sysname.
Like before, the basestation collects these packets and stores them in an SQL database for later processing and comparison with the ground truth data.
Once EnOcean detects an event and sends an occupied packet, it will not detect any more events for the next 2~minutes. 
While this 2~minute blind-spot allows the device to recharge between radio transmissions, it is also a considerable disadvantage when compared to \sysname's 6--7~second blind-spot.
\figref{fig:enoceanVwaldo} shows how the two sensors behaved in the face of a simple event and in the presence of slightly higher traffic during our deployment.


We use the data from both sensors to estimate the number of people that walk through the passageway, as shown in~\tabref{tab:ITWEnOceanVWaldoresults}.
With its smaller blindspot, \sysname outperforms the EnOcean sensors in all cases, but especially during our second deployment~(W3--W5) with increased traffic and more multi-person events.
When profiling traffic through passageways, \sysname not only provides higher resolution information, but it also provides additional direction information to building managers looking to accurately estimate traffic flows.

\begin{figure}[t]
\centering
\includegraphics[width=0.9\columnwidth]{figs/enocean_v_waldo_v_gt.pdf}
\caption{ The ground truth event timeline of our in-the-wild deployment and how \sysname and EnOcean sensors detect a sequence of events.  This example provides a high level view of each systems detection accuracy and is based on actual data collected from our in-the-wild experiments.  \label{fig:enoceanVwaldo}}
\end{figure}

\begin{table*}[t]
\footnotesize
	\begin{tabular}{@{}p{1.0in}p{0.9in}p{0.8in}p{0.7in}p{0.7in}p{0.5in}@{}}
	\toprule
	\multirow{2}{*}{\textbf{Location}}	&	\textbf{Gnd Truth} & \textbf{\sysname} & \textbf{\sysname} & \textbf{EnOcean} 	& \textbf{EnOcean}	\\
	& \textbf{Total People} & \textbf{Detected} & \textbf{Accuracy} &  \textbf{Detected} & \textbf{Accuracy} \\\midrule
	W1 & 452 & 434 & 96\% & 304 & 67\% \\ %Hallway1
	W2 & 961 & 925 & 96\% & 613 & 64\% \\ %Hallway2
	W3 & 422 & 308 & 73\% & 174 & 41\% \\	%Stairs
        W4 & 1440 & 1107 & 77\% & 551 & 38\% \\ %LabHW
	W5 & 1637 & 1316 & 80\% & 640 & 39\% \\ %preHW2
	\bottomrule
	\end{tabular}
	\caption{Comparison of performance between the EnOcean and Waldo sensors at each location during the in-the-wild deployment.  This comparison show how well each system was able to detect and monitor the number of people moving through a passageway.  Due to high traffic and burst conditions that occur when class lets out, both systems are affected with being able to detect number of people passing through the passageway as EnOcean has a 2-minute blind spot after the first event is detected and Waldo has a 6 second event window where it is processing a single event and misses multiple people traveling within that event window.
	\vspace{1mm}
	\label{tab:ITWEnOceanVWaldoresults}}

\end{table*}

%setup



%\subsubsection{Discussion of In-Wild Results.}
%These results are impacted by people lingering by doorways, entering or exiting in crowds, making abrupt changes in direction under the sensor, and by individuals changing out cameras to collect ground truth or preforming basic system maintenance.  The matched windows to draw our comparison was impacted by sporadic camera malfunctions which limited available ground truth, collection over spring break which dampened overall user activity for several days, and a school wide power-outage for system maintenance which impacted the available light to the sensors as well as the power for the basestations to collect data and log them to the SQL database.


%\fxnote{[ I created  confusion matrix using R based on the collected data, but that data needs some cleaning like i+o, (i+)o in the %detected direction-AA]}
%section:  System Accuracy
	%just walking through
		%-just using interrupts
		%-just detection circuits
		%hw detection only vs sw enhancements
	%detecting direction
		%-just using interrupts
		%-just detection circuits
		%hw detection only vs sw enhancements
%if we get to sw at all





%section:  extreme cases (how low can we make lighting and system still work)?

\section{Related Work}
\label{sec:related}

\sysname shares similarities with other occupancy-monitoring sensing systems, especially those that use doorway-mounted sensors. 
\sysname also draws from literature on sensing systems that use harvested energy both as a power source for system components and as data signals. 
Recently, a batteryless network protocol~\cite{geissdoerfer2022learning} used occupancy-monitoring as a case-study to evaluate a new network protocol.
With their solution, occupancy and direction were determined by transmitting the time between the two sensors placed adjacent to each other on the side of a doorframe when shadowed by an occupant in order to determine the direction a user traveled through the doorway.
This case study observed a small group of users on a single doorway, focusing on network performance and latency rather than how well the system performs as an occupancy detector.
This solution consumes more energy as it must transmit each time an event occurs in order to do the processing and classification off-device and may miss consecutive events due to slow charging times as occupants walk by the nodes. While this solution also using its power source as a sensor, it requires extra hardware (two nodes per doorframe) in order to perform occupancy monitoring applications. \sysname, however, uses one piece of hardware with multiple panels mounted on
top of a doorframe and processes event data on device, reducing overhead of sending timing information off device for processing, and reports multiple events to the basestation at once for energy savings, rather than needing to send each time an event occurs.  
The case-study did not provide enough information to directly compare the power-draw and performance of the two systems -- making it difficult to compare head-to-head to \sysname.  


\noindpar{Occupancy-Monitoring Systems:}
Several different methods for the detection of occupancy and inter-room movement have been explored. 
Existing occupancy monitoring systems use ultrasound\cite{hnat2012doorjamb}, imaging\cite{tyndall2016occupancy, teixeira2007lightweight}, wearables\cite{fishkin2005hands}, instrumented objects\cite{buettner2009activity}, structural vibrations\cite{pan2016occupant}, and opportunistic data leaked from existing meters and security systems\cite{yangoccupancy2014}.
These systems accurately detect occupancy (many provide other features like activity and person recognition); however, each suffers from the maintenance cost associated with battery-powered systems.

AURES~\cite{shih2016aures} attempted to address this concern by using a rechargeable battery and an indoor solar panel.
AURES estimates the number of occupants in a room by using wide-band ultrasonic signals.
It needs to be installed in a central location on the room ceiling and near a light source to function properly.
AURES, as an energy-neutral system, features an extended lifetime using energy harvesting to recharge a battery.
However, all batteries wear out (usually in a few years) meaning replacement is inevitable.
In comparison, \sysname has the dual advantage of being both easy to install (on the doorway) and batteryless (lower maintenance).

Like AURES, EnOcean~\cite{EnOcean} and Leviton~\cite{Leviton} are commercial ceiling-mounted occupancy sensors that are also powered by harvested ambient light and utilize passive infrared sensors~(PIR) for detecting occupancy through motion detection. These sensors are equipped with wireless communication capabilities for transmitting the occupancy status (occupied/not occupied) of specific rooms or areas. This is useful in controlling the lighting, HVAC, and other electrical loads. In contrast, \sysname uses the information present in harvested energy variations to detect individual doorway movements as well as the direction of those movements. This information can be used to improve utility decisions and help managers better understand how people use spaces and improve building layouts.

Another work proposes a battery-free camera powered by indoor ambient light to capture and transmit images via backscatter to a base station upon request~\cite{saffari2021battery}. 
Unlike \sysname, this system uses a duty-cycle approach rather than an event-driven one for detecting an occupant. 
This results in either higher power consumption or many missed events.

CeilingSee~\cite{yang2017ceilingsee} attempts to eliminate the extra power consumption of the monitoring tools by alternating existing LED lighting fixtures between light sources and sensors in a duty cycle manner.
It uses reflected light and machine learning to distinguish between the fixed objects in the room and the room's occupants.
CeilingSee offers a promising direction for new buildings, where custom lighting installations present an incremental cost. 
In contrast to \sysname however, applying CeilingSee to legacy installations (old buildings) would be expensive, as this would include construction costs, computational infrastructure, and IT staff maintenance.
CeilingSee could also put extra constraints on how a building can be lighted.

Recent work focuses on using multiple data sources that feed into a machine learning model to estimate the number of occupants in a building~\cite{das2017non}. 
Using the number of connected WiFi devices to detect occupant count can provide coarse-grained information; however, it's severely limited by several possible cases, such as a single occupant connecting multiple devices, use of wired internet access, or not having any device connected to WiFi. This issue is addressed by monitoring utility data, such as water and electricity consumption, weather forecast, and building functions and size along with the number of WiFi devices. This combination works well at the building level. Unlike that, \sysname is designed to monitor occupancy at room-level and communicate with other similar devices to deduce building-level occupancy. 
LOCI~\cite{narayanaloci}uses data fusion from two types of sensors PIR and thermopile to localize occupants in the workspace and estimates their height. It is not batteryless and seems to be a power-hungry system since it consumes 460mW including packet transmission. 


\noindpar{Doorway Occupancy Monitoring:}
The UVa Doorjamb sensor~\cite{hnat2012doorjamb} enabled room-level tracking of people as they moved through a house, using ultrasonic range finders mounted above a doorway, pointed towards the ground. Doorjamb differentiates people by height and detects the direction of entry and exit into the doorway. 
A recent update---SonicDoor~\cite{sonicdoor-buildsys2017}---identifies occupants by sensing their body shape, movement, and walking pattern using ultrasonic sensors embedded in the sides and top of the doorway. SonicDoor also senses user behaviors like wearing a backpack or holding a phone.
Doorjamb also used high-power sensors, wired power, and offline processing.
Both systems depend on reliable power (wired power or batteries), and use high-powered sensors (ultrasonic range finders), in contrast to \sysname, which uses energy harvesting and passive detection techniques to detect people walking through a doorway, providing room-level occupancy detection.

\noindpar{Energy as Data Sensing: } \sysname uses solar panels as both energy source and sensor simultaneously. This technique has been used in other systems for applications other than occupancy monitoring. Monjolo~\cite{debruin2013monjolo} measures the AC load consumption based on the harvested power from the AC load. Trinity~\cite{xiang2013powering} is designed to measure the airflow speed of air-conditioning based on the harvested power from piezoelectricity that is generated from the impact of airflow. DoubleDip~\cite{martin2012doubledip} adapted this technique to monitor the water flow through a pipe using a thermoelectric generator as a harvester and sensor. Along with these, KEH-Gait~\cite{xu2017keh} is designed for healthcare authentication and providing activity tracking. It does this by sensing the voltage level produced by two types of kinetic harvesters (piezoelectric and electromagnetic), which simultaneously also power the system. There has been another attempt to design a battery-free pedometer~\cite{kalantarian2016pedometers} by placing a piezoelectric harvester inside a shoe and estimating the number of steps based on the amount of harvested energy. 

In addition, some indoor-sensing and ambient light-powered systems utilize solar panels as either a power source or sensor~\cite{campbell2014energy, billah2022solarwalk, li2018self}, but not both. SolarWalk~\cite{billah2022solarwalk} does use ambient light and a solar panel as its sensor, but doesn't harvest energy from the panel to power the system.  It uses a PIR sensor to detect when a person is crossing the threshold and then records and transmits the raw solar panel data to be processed off-device for classification and identification of subjects.  While it does point to an exciting direction of identifying users using solar data, it is limited in it's deployability as it is wall-powered, energy-expensive having to send off raw data each time there is an event, and processes the data off-device, which can expose privacy risks.
SolarGest~\cite{ma2019solargest} and \sysname share a similar method for using solar energy as an indicator for light ray interference, SolarGest is also batteryfree. However, SolarGest differs from \sysname in application, implementation, and focus. SolarGest measures a small set of gestures (6) that are performed close to the solar panel; these controlled conditions are in contrast to \sysname, which must deal with a wide range of confounding conditions and scenarios. Unlike SolarGest, \sysname does all recognition in-situ, while SolarGest must rely on a backscatter communication channel with which it sends all data. This offline processing severely constrains the applicability of SolarGest. Li et al.~\cite{li2018self} used photodiodes for both energy harvesting and sensing to recognize finger gestures on wrist- and head-worn wearables. In addition to significant application differences, they use a traditional duty-cycled ADC-based design that results in significantly-higher power consumption and requires more computation and orders of magnitude more energy storage (a \SI{0.22}{\farad} supercapacitor compared to \sysname's \SI{100}{\micro\farad} capacitor). Using this approach for doorway event detection would require a significantly larger prototype. 
Also, unlike \sysname, this work does not allow an energy channel to be used for harvesting and sensing at the same time for midair swipe-gestures---each photodiode unit is periodically disconnected from the harvestor circuit when needed for measurements.
\sysname's detector circuits enable simultaneous detection and harvesting in order to improve harvesting efficiency.



\noindpar{Batteryless, Transiently Powered Sensing: }
Recent work like InK~\cite{yildirim2018ink}, HarvOS~\cite{bhatti2017harvos}, Mayfly~\cite{hester2017mayfly}, and Ratchet~\cite{van2016intermittent} have explored operating system and language-level support for developing applications easily on batteryless devices with frequent power failures.
Others have focused on energy management and storage techniques, like  Federated Energy~\cite{jhester:ufop:sensys}, to improve system uptime and responsiveness.
These systems inform our work, however, none has tackled the problem of batteryless occupancy monitoring.

\section{Discussion \& Future Work}
\label{sec:discussion}

In this paper, we demonstrated that we can monitor how people use buildings without running wires, without structural renovations, and without batteries.
Our evaluation has presented the performance of \sysname as a batteryless occupancy sensor and we also have identified corner cases that might confound the current version of the system.
This section describes our future plans in terms of making \sysname more robust and reliable.
We also present some ideas for expansion of this project.

\noindpar{Improving robustness and reliability:} \sysname in its current version depends on sudden changes in the solar panel outputs in a fairly binary manner.
It triggers when there's change and doesn't when there isn't.
This allows it to detect people walking through with high accuracy.
However, it becomes susceptible to false positives as other events might also cause a sudden change in the solar panel, for example when someone walks by the side of the door.
As discussed in \secref{sec:evaluation}, there is a visible difference between someone walking through a doorway and a false positive.
One of our goals for future work is to add direct signal processing on the microcontroller so that it will be able to access the whole shape of the waveform, and will not be reliant on the binary nature of the detector interrupts.
This will enable \sysname to be more successful in identifying people walking through the doorway with minimal false positives.

\noindpar{User perceptions of privacy:}
Occupancy monitoring is often privacy violating---cameras, audio, and other methods being examples.
Privacy rights in the workspace have long been debated~\cite{oz1999electronic}, with some workers reporting productivity suffered because of the perception of loss of privacy~\cite{stanton2000electronic}.
Even though \sysname is privacy preserving, and incapable of gathering video, audio, or other personal information, we have not yet surveyed people who live and work with \sysname in their room or office.
We believe user perceptions of their privacy could inform both the design of future \sysname prototypes and provide insight into this tension between privacy and real time occupancy monitoring.
We plan to explore this in future work.


\noindpar{Adaptability:} We plan to make the system more dynamic and flexible by providing adjustable thresholds to the detector circuit.
This will equip \sysname with the ability to tune its sensitivity to problematic cases, such as darker lighting conditions.
Another way we aim to improve the performance and adaptibility of \sysname would be to make use of learning algorithms.
Our goal is to use learning for identifying different events and separating the true positives from false ones, subsequently improving accuracy and precision.
We will introduce confidence indicators so that, even in cases where it is comparatively tougher to distinguish between those events, \sysname will be able to attach a confidence level to its prediction, broadening the range of events it can identify.
This is a feasible goal considering the evident difference between those events.

\noindpar{Network of Waldos:}
\sysname is not meant to be a standalone system in that its true potential is realized as a part of a wider network of similar sensors.
Different Waldos could exchange information to monitor occupancy on a larger scale and also to improve individual performance.
For example, if one sensor detects a large amount of traffic heading into a hallway, but none of the other sensors detect activity, it is likely that there might be some other factor that is confounding the first sensor and this knowledge could be used to refine the learning model.
Having a network of such batteryless sensors could also enable the deployment of a more sophisticated, energy-efficient communication model than simply broadcasting information opportunistically.

\noindpar{Additional sensors:} We also plan to expand the system in terms of sensing abilities by adding more sensors.
These sensors could provide various types of information such as RGB data, which could be used to semi-identify the person walking through.
This would help assign some uniqueness to each individual so that we can better track their travel through rooms in a building without gathering identifiable information that would require additional security considerations to be added to the system.
We could also opportunistically use an ultrasonic range finder in moments of high illumination to detect the height of the person passing through.

\sysname can be expanded in many different ways, as demonstrated by these ideas.


%Communication-
%currently all data is stored locally on the board and ideally we would want to be able to send the data to a basestation or central location to gain the overall benefit of making use of the room-level information that the \sysname sensor is collecting.  A network of these sensors would be necessary in a true deployment to gain a real understanding of the users and their room usage.  We are continuing to develop the system to allow for this greater flexibility and usefulness.  \fxnote{[Something like this??? -NT]}

%Alternative sensors--
%\sysname currently only uses an array of solar panels to collect energy and to gather the information necessary to determine if a person has passed through the doorway.  The system's accuracy and functionality could be improved by using additional and/or alternative sensors such as ultrasonic and RGB light sensors.  Detection is evaluated in our system only from the change in the signals that the solar panels are receiving. A change in typical color could not only be an indication of someone passing through the doorway but also a means of helping to semi-identify the person walking through.  This would help to determine some uniqueness of the individuals to better track their travel through rooms in a building without gathering identifying information that would require additional security considerations to be added to the system.  \fxnote{[Not sure which other sensors to reference-NT]}

%height estimation ---
%this can be accomplished with the ultrasonic sensors and possibly with the solar panels currently in the system alone... \fxnote{Need to elaborate with this- NT}

%adaptation --- as environments change so will energy budgets.
%We take a simple approach, that we expect to work for most indoor environments; however, within these environments there are still a great different in lighting
%We plan to explore techniques for adapting sensing modalities to match current budgets and conditions.

%recognition

%\fxnote{Not sure where to talk about it. Maybe in System Design (or here), but we should address the question about other orientations. We could put \sysname on the side of the doorway and possibly harvest more energy. Or in the floor. In the floor, it would be harder to deploy, and people would step on it, but then you break the beam directly. Lots more energy, more wear and tear, possibly more difficult signal processing from different light configurations...people we just didn't do it. Maybe we will soon.}

\section{Conclusions}
\label{sec:conclusions}

This paper has presented \sysname, a batteryless, energy harvesting doorway mounted sensor system for room level occupancy monitoring.
\sysname uses its energy source---generated from an array of solar cells---as data signal for detecting doorway movement, as well as the energy that powers all activities.
\sysname uses a novel, tunable, detection circuit that watches the energy harvesting signal with the processor asleep, and it is the first batteryless occupancy monitoring system in existence.
We deployed \sysname on \numDoors doorways and found that it can detect single persons moving through the doorways with a high overall detection accuracy of \SysAccuracy. Our results show that \sysname can differentiate between entry and exit of persons walking through the doorway for \dirAccuracy of the detected events.
We also evaluated different factors that affect the performance of \sysname.
There are some events that confound the current version of \sysname into generating false positives, but we have demonstrated inherent differences from the true positives \ie someone walking through the door.
This makes us confident that we can further improve the \sysname system to make it robust to such events.
We evaluated \sysname microbenchmarks that demonstrate \sysname is low power, and efficient, able to harvest enough energy to power all activities, intermittently, while providing quality of application.
\sysname represents a first step towards robust and reliable occupancy monitoring systems without batteries, using energy harvesting.

% ACKs only appear if anonymous flag is false for acmart.cls
\begin{acks}
This research is supported by NSF award CNS-XXXXXXX.%1453607. 
Any opinions, findings, conclusions, and recommendations expressed in these
materials are those of the authors and do not necessarily reflect the
views of the sponsors.
\end{acks}
%-------------------------------------------------------------------------------
\begin{artifacts}
%The artifacts associated with this paper are the PCB design and 3D prototype housing files used to create the sensor presented in this work as well as the software for training and implementing the deployed \sysname functionality. 
The artifacts associated with this paper consist of both hardware and software files needed for building a functioning deployment of the Ray sensor.  We are including the custom PCB designs, Bill of Materials, and 3D prototype housing files used to create a Ray sensor. PDFs of the schematic and board design files are also included for reviewer convenience.  Software for training and the firmware that runs the Ray sensor and base station are also provided.  Documentation includes both how to build, set up, tune, train, and deploy a Ray sensor as well as how to set up a base station for receiving the sensor data.
\end{artifacts}

%-------------------------------------------------------------------------------

%-------------------------------------------------------------------------------
\bibliographystyle{plainurl}
\bibliography{persist}

%%%%%%%%%%%%%%%%%%%%%%%%%%%%%%%%%%%%%%%%%%%%%%%%%%%%%%%%%%%%%%%%%%%%%%%%%%%%%%%%
\end{document}
%%%%%%%%%%%%%%%%%%%%%%%%%%%%%%%%%%%%%%%%%%%%%%%%%%%%%%%%%%%%%%%%%%%%%%%%%%%%%%%%

%%  LocalWords:  endnotes includegraphics fread ptr nobj noindent %  LocalWords:
%pdflatex acks
