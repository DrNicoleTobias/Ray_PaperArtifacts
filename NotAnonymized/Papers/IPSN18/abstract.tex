The buildings of our science fiction dreams have always adapted to the needs of their occupants.
Today, "smart buildings" are poised to become reality, enabled by advances in sensors that monitor room-level occupancy and movement.
Unfortunately, existing occupancy-tracking systems are plagued by large size, high energy consumption, and, unsurprisingly, short battery lifetimes.

In this paper, we present \sysname, a \textit{batteryless}, room-level occupancy monitoring sensor that harvests energy from indoor ambient light reflections, and uses changes in these reflections to detect when people enter and exit a room.
%This information is then communicated by radio to a basestation for further processing and actuation.
Like previous systems, \sysname is mountable at the top of a doorframe, allowing for detection and tracking of a person at the entry and exit point of a room.
We evaluated the \sysname sensor in an office-style setting using both ambient and traditional fluorescent lighting sources on both sides of the doorway with subjects exhibiting varying physical characteristics such as height, hair color, gait, and clothing.
While challenges remain, this work demonstrates that ambient light reflections provide both a promising low-cost, long-term sustainable option for monitoring how people use buildings and an exciting new research direction for \textit{batteryless} computing.
