\section{Related Work}
\label{sec:related}

\sysname is closely related to other occupancy monitoring sensing systems---especially those using doorway mounted sensor suites. 
\sysname also draws from the literature on sensing systems that treat harvested energy as energy and data signal; deriving application information from the energy source. 
We detail the related work to \sysname below.

\noindpar{Occupancy Monitoring Systems:} 
Many exotic methods for occupancy detection, and movement between rooms have been explored. Existing occupancy monitoring systems use ultrasound\cite{hnat2012doorjamb}, imaging\cite{tyndall2016occupancy, teixeira2007lightweight}, wearables\cite{fishkin2005hands}, instrumented objects\cite{buettner2009activity}, structural vibrations\cite{pan2016occupant}, and opportunistic data leaked from existing meters and security systems\cite{yangoccupancy2014}. 
Each of these systems proved accurate in occupancy detection (and often provided further features such as activity and person recognition), however, each suffered from the maintenance cost associated with battery powered systems.
These monitoring systems did not address the issue of the power source, as it is supposed to eliminate the power load of the building and laborious work of replacing the battery for that type of device.
AURES~\cite{shih2016aures} attempted to address this concern by using a rechargeable battery and a indoor solar panel. 
AURES estimates the number of occupants in a room by using wide-band ultrasonic signals. It needs to be installed in a central location on the room ceiling and near a light source to function properly.
AURES, as an energy-neutral system, features an extended lifetime using energy harvesting to recharge a battery, however, all batteries wear out (usually in a few years) meaning replacement is inevitable. Unlike \sysname, AURES is not installed on the doorjamb and is not a batteryless and maintenance free device.
Ceilingsee~\cite{yang2017ceilingsee} attempts to eliminate the extra power consumption of the monitoring tools by customizing room lighting�using LEDs as a light source and sensors in a duty cycle manner. It uses reflected light and machine learning to distinguish between the fixed objects in the room and the room?s occupants, unlike  \sysname that senses and harvests power simultaneously. CeilingSee offers a promising direction for new buildings, where custom lighting installations present an incremental cost. Applying CeilingSee to legacy installations (old buildings) would be expensive, as this would include construction costs, computational infrastructure, and IT staff maintenance. CeilingSee may also put extra constraints on how a building can be lighted.
Recent work focuses on using multiple data sources that feed into a machine learning model to estimate the number of occupants in a building~\cite{das2017non}. The number of WiFi devices is not enough to monitor where one occupant may have multiple devices, using wired internet access, or not having any device. This work eliminates this gap by monitoring utility data, such as water and electricity consumption, weather temperature, and building functions and size along with the number of WiFi devices. This work focuses on building occupancy monitoring and is unlike \sysname, which is designed to monitor occupancy at room level. Despite this approach, installing any hardware is not required. It instead requires training in the occupancy estimation model for each building individually due to the differences in buildings characteristics that affect the trained model.


\noindpar{Doorway Occupancy Monitoring:} 
Closely related to \sysname are doorway occupancy monitoring systems; the UVa Doorjamb sensor being the first significant work~\cite{hnat2012doorjamb}.
UVa Doorjamb enabled room level tracking of people as they moved through a house, by way of ultra sonic range finders mounted in the top of the doorway, pointing towards the ground. Doorjamb could differentiate people by height, and detect direction of entry and exit into the doorway. Doorjamb was plugged into an outlet, and used high power sensors to gather data, which was processed later.
Recently, SonicDoor~\cite{sonicdoor-buildsys2017}---an update to Doorjamb---was developed which identifies occupants by sensing their body shape, movement and walking pattern using ultrasonic ping sensors embedded in the sides, and top of the doorway. SonicDoor also senses user behaviors such as wearing a backpack or holding a phone.
Both of these techniques use reliable power or batteries and high powered sensors like the ultra sonic range finder, \sysname uses energy harvesting and batteryless, passive detection techniques to detect people walking through a doorway, informing occupancy detection. 

\noindpar{Energy as Data Sensing: } \sysname uses solar panels as the energy source and a sensor at the same time. This technique has been used in other systems rather than occupancy monitoring. Monjolo~\cite{debruin2013monjolo} measures the AC loads consumption based on the harvested power from the AC load. Also, Trinity~\cite{xiang2013powering}is designed to measure the airflow speed of air-conditioning based on the harvested power from piezoelectricity that generated from the impact of air flow. DoubleDip~\cite{martin2012doubledip} is another monitoring system that adapted this technique to monitor the water flow through a pipe using thermoelectric generator as a harvester and sensor.
Despite the fact that there is an indoor-sensing architecture that uses indoor solar-harvested power~\cite{campbell2014energy}, this architecture does not support the idea of using the energy harvesting source as the data sensor, as used in \sysname. 
\sysname is the first batteryless energy harvesting occupancy monitoring platform, that builds on other works by gathering energy and data from the harvester. 

\noindpar{Batteryless, Transiently Powered Sensing: }
\sysname is not the first batteryless, transiently powered sensing system.
Recent work like HarvOS~\cite{bhatti2017harvos}, Mayfly~\cite{hester2017mayfly}, and Ratchet~\cite{van2016intermittent} have explored operating system and language level support for developing applications easily on batteryless devices with frequent power failures.
Other work has focused on energy management and storage techniques to improve uptime and responsiveness of these systems: such as Federated Energy~\cite{jhester:ufop:sensys}.
Each of these systems inform our work, however, none have tackled the problem of batteryless occupancy monitoring. 