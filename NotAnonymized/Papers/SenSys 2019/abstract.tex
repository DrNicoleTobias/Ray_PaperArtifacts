%The buildings of our science fiction dreams have always adapted to the needs of their occupants.
%Today, "smart buildings" are poised to become reality, enabled by advances in sensors that monitor room-level occupancy and movement.
Reliable and accurate room-level occupancy-tracking systems enable intelligent control of building functions like air conditioning and power delivery to adapt to the needs of their occupants.
%This allows buildings to be more capable of adapting to the needs of their occupants in their day-to-day activities and better optimize certain resources, such as power and air conditioning, to do so.
Unfortunately, existing occupancy-tracking systems are bulky, have short battery lifetimes, or are not privacy preserving.
Furthermore, retrofitting existing infrastructures with wired sensors is prohibitively expensive.

In this paper, we present \sysname, a \textit{batteryless}, room-level occupancy monitoring sensor that harvests energy from indoor ambient light reflections, and uses changes in these reflections to detect when people enter and exit a room.
%This information is then communicated by radio to a basestation for further processing and actuation.
\sysname is mountable at the top of a doorframe, allowing for detection of a person and the direction they are traveling at the entry and exit point of a room.
We evaluated the \sysname sensor in an office-style setting under mixed lighting conditions (natural and artificial) on both sides of the doorway with subjects exhibiting varying physical characteristics such as height, hair color, gait, and clothing.
We conducted \numExp controlled experiments in \numDoors doorways with \numPeople individuals and achieved a total detection accuracy of \SysAccuracy and movement direction accuracy of \dirAccuracy.  
We also describe an uncontrolled in-the-wild experiment using \sysname sensors across \ITWdays collectively in three different locations.
%Further, it judged the direction of movement correctly with an accuracy of \dirAccuracy.
%This paper also explores and discusses various practical factors that can impact the performance of the current system in actual deployments.
%Finally, we evaluate the power consumption of the \sysname device, demonstrating it's low-power draw.
%While challenges remain, t
\sysname demonstrates that ambient light reflections provide both a promising low-cost, long-term sustainable option for monitoring how people use buildings and an exciting new research direction for \textit{batteryless} computing.
