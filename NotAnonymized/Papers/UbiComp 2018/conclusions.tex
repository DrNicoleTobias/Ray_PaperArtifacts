\section{Conclusions}
\label{sec:conclusions}

This paper has presented \sysname, a batteryless, energy-harvesting doorway-mounted sensor system for room level occupancy monitoring.
\sysname uses its energy source---generated from an array of solar cells---as data signal for detecting doorway movement, as well as the energy that powers all activities.
\sysname uses a novel, tunable, detection circuit that watches the energy harvesting signal with the processor asleep, and it is the first batteryless occupancy-monitoring system in existence.
We deployed \sysname on \numDoors doorways and found that it can detect single persons moving through the doorways with a high overall detection accuracy of \SysAccuracy. Our results show that \sysname can differentiate between entry and exit of persons walking through the doorway for \dirAccuracy of the detected events.
We also evaluated different factors that affect the performance of \sysname.
There are some events that confound the current version of \sysname into generating false positives, but we have demonstrated inherent differences from the true positives \ie someone walking through the door.
This makes us confident that we can further improve the \sysname system to make it robust to such events.
We evaluated \sysname microbenchmarks that demonstrate \sysname is low power, and efficient, able to harvest enough energy to power all activities, intermittently, while providing quality of application.
\sysname represents a first step towards robust and reliable occupancy-monitoring systems without batteries, using energy harvesting.
