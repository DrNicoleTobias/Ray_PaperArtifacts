\section{Discussion \& Future Work}
\label{sec:discussion}

In this paper, we demonstrate that we can monitor how people use buildings without running wires, requiring structural renovations, and without batteries.
Our results are, to date, both promising and limited.
This section describes both our current limitations and our future plans.

\subsection{Limitations}
Limitations of \sysname are exhibited because of variability of people, and rooms in buildings. 
We will never be able to test every combination of these variables; however, in the coming months, we plan to expand our experiments to include more participants, a wider range of environments, and lighting conditions. We list confounding factors below:

%refine once it's clear what we actually looked at.
\noindpar{Lighting Conditions:} Ambient and florescent light reflections provide convenient information and energy sources; however, the lighting conditions that \sysname will face when deployed is difficult to predict at design time.
Our experiments have explored a typical florescent lighting scenario encountered in an office setting but this does not account for the impact of additional lighting sources such as windows and multiple lighting sources in a room or lack of lighting sources (such as a home with no overhead lighting).  
With rooms that include windows, time of day could have a great impact on the energy available to the system and light available to detect individuals.  
%Rooms that are mainly dependent on lamps and other non-overhead lighting may provide less light to make accurate detection and decisions based on those detections.
%Another impact to the accuracy of the system may include lighting variation wherein one room may have a significantly different level of light than the adjacent room.

\noindpar{Flooring:} \sysname was tested in an environment with a semi-reflective tile floor.  
In an office building or home, there maybe many different types of flooring encountered by the sensor.  
The diffuse and reflective attributes of materials would help or hinder the signal and amount of energy harvested by \sysname.
Some doorways may also experience a mix of flooring types like those  between a transition from a carpeted room to a laminate floored kitchen.

\noindpar{Human variation:} Differences in skin/clothing color, height, gait, and body size did not noticeably affect \sysname's accuracy for our well-lit doorway.  However, it is not know how much these factors influenced the detection of direction in the second doorway, that was not as well lite on one side.
Additionally, readings from a single individual walking through the doorway will produce a different signal than a group of people walking through a doorway together or in rapid succession as would happen when a meeting lets out.

\noindpar{Problematic cases:} Fortunately, doorways tend to be used for their intended purpose and \sysname can improve its accuracy from anticipating those typical actions.  However, there are some problematic cases for this system that happen from time-to-time by human users.  These include actions such as lingering in a doorway, poking a head in for a brief moment, or passing close by the doorway of interest.  These problematic cases can produce false positives in the system as they may still impact the readings from the sensor. 


\subsection{Future Work}
For occupancy sensing, it is observed that it is much better to enable longer streams of uninterrupted sensing at a lower application quality or application accuracy than to gather short, intermittent bursts at a higher quality, requiring more energy, and constraining execution to only a certain level of available energy.
To support nearly continuous sensing, approximate computing can be leveraged, trading accuracy and quality of services for uninterrupted operation.
The following are required for this to work: 1)~identifying the levels of service that can be supported by the application, and 2)~knowing when to switch between these levels of service.
Approximate computing applied to batteryless systems makes intuitive sense when sensor data exhibits high temporal locality, meaning that a data point gathered immediately after another may be worthless, as nothing has changed (for example when monitoring an empty doorway).
Knowing when to reduce the level of service is a key challenge that we plan to explore in future work.
This approximation approach could also benefit from integrating other sensors; such as RGB color sensors or ultra sonic range finders.
These sensors could help differentiate or identify persons walking through the doorway (using either height or relative color).

%Communication-
%currently all data is stored locally on the board and ideally we would want to be able to send the data to a basestation or central location to gain the overall benefit of making use of the room-level information that the \sysname sensor is collecting.  A network of these sensors would be necessary in a true deployment to gain a real understanding of the users and their room usage.  We are continuing to develop the system to allow for this greater flexibility and usefulness.  \fxnote{[Something like this??? -NT]}

%Alternative sensors--
%\sysname currently only uses an array of solar panels to collect energy and to gather the information necessary to determine if a person has passed through the doorway.  The system's accuracy and functionality could be improved by using additional and/or alternative sensors such as ultrasonic and RGB light sensors.  Detection is evaluated in our system only from the change in the signals that the solar panels are receiving. A change in typical color could not only be an indication of someone passing through the doorway but also a means of helping to semi-identify the person walking through.  This would help to determine some uniqueness of the individuals to better track their travel through rooms in a building without gathering identifying information that would require additional security considerations to be added to the system.  \fxnote{[Not sure which other sensors to reference-NT]}

%height estimation ---
%this can be accomplished with the ultrasonic sensors and possibly with the solar panels currently in the system alone... \fxnote{Need to elaborate with this- NT} 

%adaptation --- as environments change so will energy budgets.
%We take a simple approach, that we expect to work for most indoor environments; however, within these environments there are still a great different in lighting 
%We plan to explore techniques for adapting sensing modalities to match current budgets and conditions.

%recognition

%\fxnote{Not sure where to talk about it. Maybe in System Design (or here), but we should address the question about other orientations. We could put \sysname on the side of the doorway and possibly harvest more energy. Or in the floor. In the floor, it would be harder to deploy, and people would step on it, but then you break the beam directly. Lots more energy, more wear and tear, possibly more difficult signal processing from different light configurations...people we just didn't do it. Maybe we will soon.}
